\begin{doublespace}
  \begin{tightcenter}
    \section{4. Objetivos}
    \mylinespacing
  \end{tightcenter}

  \subsection{4.1  Objetivo General}

  Crear un servicio de computación a gran escala con capacidades de paralelización, accesible a la comunidad universitaria, dispuesto para la investigación, a partir del aprovechamiento de recursos subutilizados existentes en el Departamento de Matemáticas de la Universidad del Valle.

  \subsection{4.2  Objetivos Específicos}

  \begin{enumerate}
    \item Identificar recursos disponibles y necesidades investigativas.
    \item Diseñar una solución que considere los requerimientos de los usuarios y aproveche las capacidades de los recursos computacionales mediante la implementación de un gestor de cola de tareas.
    \item Llevar a cabo la instalación, documentación y puesta a punto de herramientas que apoyen los procesos de investigación y docencia en el área de matemáticas, facilitando el uso del servicio.
    \item  Llevar a cabo pruebas de uso de la infraestructura y de las aplicaciones desplegadas en este trabajo.
  \end{enumerate}

  \let\cleardoublepage\clearpage
  \newpage
  \subsection{4.3 Resultados Esperados}

  \begin{table}[ht]
    \centering
    \caption{Productos Esperados}
    \begin{tabular}{p{7cm}p{7cm}}
      \hline
      \centering\textbf{Objetivos Específicos}                                                                                                                                                           & \textbf{Producto(s) Esperados}                                                                                                                                  \\
      \hline
      \text Identificar recursos disponibles y necesidades investigativas.                                                                                                                               & Documento expresando la arquitectura de los recursos y las necesidades a considerar.                                                                            \\
      \hline
      \text Diseñar una solución que considere los requerimientos de los usuarios y aproveche las capacidades de los recursos computacionales mediante la implementación de un gestor de cola de tareas. & Documentación, scripts, programas y pruebas básicas de operación que simplifiquen o automaticen el mantenimiento y administración del gestor de cola de tareas. \\
      \hline
      \text Llevar a cabo la instalación, documentación y puesta a punto de herramientas que apoyen los procesos de investigación y docencia en el área de matemáticas, facilitando el uso del servicio. & Documentación, scripts y programas que automaticen y faciliten la utilización del servicio para los estudiantes o profesores.                                   \\
      \hline
      \text Llevar a cabo pruebas de uso de la infraestructura y de las aplicaciones desplegadas en este trabajo.                                                                                        & Reporte de pruebas en donde se evidencie la correcta funcionalidad y la eficiencia de los recursos.                                                             \\
      \hline
    \end{tabular}
    \label{table:table1}
  \end{table}

  \subsection{4.4 Alcances de la Propuestas}

  Este proyecto surge de la visión de León Escobar, Co-director de este proyecto, quien identificó una necesidad de computo en algunos trabajos de investigación que no es accesible actualmente y como profesor a cargo de la sala de Cómputo Jürgen Tischer visibilizó una posible solución con recursos ya disponibles mediante la computación distribuida.

  \vspace{3mm}

  En este contexto, el alcance del proyecto se limita a diseñar una estrategia que permita la utilización de los recursos de cómputo pertenecientes al departamento de matemáticas de la Universidad del Valle, en paralelo mediante un gestor de cola de tareas y llevar a cabo dicha estrategia teniendo como prioridad la facilidad de uso para el usuario, teniendo en cuenta las necesidades comunes para matemáticas en la investigación.


  \mylinespacing
  \mylinespacing
  \begin{tightcenter}
  \end{tightcenter}
\end{doublespace}
