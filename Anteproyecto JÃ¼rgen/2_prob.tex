\begin{doublespace}
\begin{tightcenter}
\section{2. Planteamiento del problema}
\mylinespacing
\end{tightcenter}

\subsection{2.1  Descripción del problema}

Actualmente los trabajos investigativos en el área de ciencias de la Universidad del Valle son principalmente teóricos, sin grandes requerimientos computacionales, esto no es debido a una falta de trabajos que requieran gran poder de procesamiento computacional, sino debido a una carencia de acceso a estas capacidades computacionales para suplir las necesidades.

\vspace{3mm} 

Esta idea se presenta de la visión y necesidad de ejecutar cálculos matemáticos a gran escala, con propósitos investigativos, de parte del departamento de Matemáticas.

\vspace{3mm} 

\subsection{2.2 Formulación del Problema}

¿Cómo se podría suplir la necesidad de computación necesaria para investigación matemática con los recursos disponibles?

\vspace{2mm} 

¿Es posible hacer que estos recursos sean accesibles y fáciles de usar, es decir, que no requieran de conocimientos técnicos elevados para ser aprovechables por la comunidad universitaria?





\mylinespacing
\mylinespacing
\begin{tightcenter}
\end{tightcenter}
\end{doublespace}
