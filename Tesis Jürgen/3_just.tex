\begin{doublespace}
\begin{tightcenter}
\section{3. Justificación del Problema}
\mylinespacing
\end{tightcenter}

\subsection{3.1 Justificación Académica}

Con una capacidad de procesamiento a gran escala con capacidades de paralelización se abren las puertas a nuevos trabajos investigativos, no solamente para el departamento de matemáticas sino también para toda el área de Ciencias de la Universidad.

\subsection{3.2  Justificación Económica}

Existen medios similares a los propuestos por este trabajo, que normalmente se encuentran en la nube, pero no existen los recursos económicos para ser utilizados por estudiantes. Este proyecto busca aprovechar recursos subutilizados dentro de la Universidad del Valle, por lo que al repotenciar estos elementos existentes, se ahorra en la inversión de nuevos bienes a la vez que se hacen accesibles para los estudiantes y profesores investigadores.

\subsection{3.2  Justificación Económica}

Poseer un recurso propio de bajo costo permite la democratización de acceso a plataformas de alto rendimiento o masivamente paralelas.

\mylinespacing
\mylinespacing
\begin{tightcenter}
\end{tightcenter}
\end{doublespace}
