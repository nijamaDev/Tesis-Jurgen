\begin{doublespace}
  \begin{tightcenter}
    \section{6. Metodología}
    \mylinespacing
  \end{tightcenter}

  \subsection{6.1 Actividades a Realizar}

  \begin{table}[ht]
    \centering
    \caption{Actividades y resultados}
    \begin{tabular}{p{4cm}p{5cm}p{4cm}}
      \hline
      \centering\textbf{Objetivos Específicos}                                                                                                                                                            & \centering\textbf{Actividad}                                                                                                                                    & \textbf{Resultado Esperado por Actividad}                                            \\
      \hline

      \vspace{1mm}
      \text Identificar recursos disponibles y necesidades investigativas.                                                                                                                                & Identificar la totalidad de los recursos disponibles y las necesidades investigativas para el proyecto.                                                         & Documento expresando la arquitectura de los recursos y las necesidades a considerar. \\
      \hline Diseñar una solución que considere los requerimientos de los usuarios y aproveche las capacidades de los recursos computacionales mediante la implementación de un gestor de cola de tareas. & Identificación e implementación del gestor de cola de tareas adecuado para la arquitectura y las necesidades

      \vspace{5mm}

      Identificación e implementación del gestor de cola de tareas adecuado para la arquitectura y las necesidades.                                                                                       & Documentación, scripts, programas y pruebas básicas de operación que simplifiquen o automaticen el mantenimiento y administración del gestor de cola de tareas.                                                                                        \\
      \hline
      \text Llevar a cabo la instalación, documentación y puesta a punto de herramientas que apoyen los procesos de investigación y docencia en el área de matemáticas, facilitando el uso del servicio.  & Identificación de puntos de automatización.

      \vspace{5mm}

      Integración de herramientas de apoyo según las necesidades identificadas o creación de nuevas soluciones en caso de ser necesario.                                                                  & Documentación, scripts y programas que automaticen y faciliten la utilización del servicio para los estudiantes o profesores.                                                                                                                          \\
      \hline
      \text Llevar a cabo pruebas de uso de la infraestructura y de las aplicaciones desplegadas en este trabajo.                                                                                         & Reporte de pruebas en donde se evidencie la correcta funcionalidad y la eficiencia de los recursos.                                                                                                                                                    \\
      \hline
    \end{tabular}
    \label{table:table1}
  \end{table}

  \mylinespacing
  \mylinespacing
  \begin{tightcenter}
  \end{tightcenter}
\end{doublespace}
