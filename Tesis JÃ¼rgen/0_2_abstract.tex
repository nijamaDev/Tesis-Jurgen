\begin{spacing}{1.5}
\begin{abstract}
    El presente trabajo tiene como objetivo proponer una metodología para aprovechar al máximo los recursos computacionales subutilizados en el clúster Bochica y la sala de computación Jürgen Tischer del departamento de Matemáticas de la Universidad del Valle. Se propone la implementación de un sistema distribuido con un gestor de cola de tareas y capacidades de paralelización para lograr una mayor eficiencia en investigaciones matemáticas. Se investiga y se propone una implementación específica del sistema distribuido y se evalúa su rendimiento y eficiencia en comparación con otros métodos existentes. Los resultados obtenidos muestran una mejora significativa en el aprovechamiento de los recursos computacionales y una mayor eficiencia en las investigaciones matemáticas. Este trabajo contribuye al campo de la informática y la matemática aplicada, al mejorar el aprovechamiento de los recursos computacionales en entornos académicos.\end{abstract}
\end{spacing}
