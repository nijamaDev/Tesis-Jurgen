\begin{doublespace}
\begin{tightcenter}
\section{1. Introducción}
\mylinespacing
\end{tightcenter}

\begin{multicols}{2}

    \subsection{1.1  Introducción}

    El objetivo de la tesis presentada es maximizar la utilización de los recursos disponibles en el clúster Bochica y la sala de cómputo Jürgen Tischer del departamento de Matemáticas de la Universidad del Valle. A pesar de contar con una cantidad considerable de medios computacionales, estos recursos no están siendo utilizados de manera óptima. Para abordar este problema, se propone una metodología que consiste en la implementación de un gestor de cola de tareas con capacidades de paralelización, lo que permitirá una utilización más eficiente y efectiva de los recursos en investigaciones matemáticas.
    \vspace{3mm} 

    Para alcanzar este objetivo, se propone utilizar sistemas distribuidos como estrategia para aprovechar al máximo los recursos disponibles. Estos sistemas permiten la paralelización de tareas y la distribución de la carga de trabajo entre varios nodos o dispositivos, lo que implica la posibilidad de aprovechar la capacidad total de cómputo de los recursos disponibles. Además, estos sistemas ofrecen una mayor flexibilidad y escalabilidad en el uso de los recursos, lo que resulta esencial en un entorno de investigación matemática donde las necesidades de cómputo pueden variar significativamente.
    \vspace{3mm} 

    Se realizará un análisis de diversos programas informáticos en esta tesis con el fin de proponer una implementación específica de un sistema distribuido que permita un mayor rendimiento y eficiencia en comparación con otros métodos existentes. Además, se evaluará el desempeño del sistema propuesto en relación con otros métodos, para determinar su efectividad en la optimización del uso de los recursos disponibles en el clúster Bochica y la sala de computación Jürgen Tischer.
    
    \subsection{1.2 Planeamiento del problema}

    \textbf{Descripción del problema}
    \vspace{3mm}

    La falta de acceso a capacidades computacionales adecuadas está limitando la capacidad de los trabajos de investigación en el campo de las ciencias en la Universidad del Valle para abordar problemas complejos, realizar análisis de datos y simular procesos. Esto se ha reflejado en un enfoque predominantemente teórico en lugar de computacionalmente intenso en estos trabajos de investigación. El Departamento de Matemáticas identifica esta limitación como un problema y ve la necesidad de acceder a recursos computacionales más potentes para poder llevar a cabo cálculos matemáticos a gran escala con fines investigativos.
    \vspace{3mm}

    \textbf{Formulación del problema}
    \vspace{3mm}

    ¿Cómo podrían satisfacerse las demandas de cómputo requeridas para la investigación matemática utilizando los recursos disponibles de manera eficiente? ¿Cómo se podría facilitar el acceso y uso de estos recursos para que puedan ser utilizados por la comunidad universitaria sin necesidad de conocimientos técnicos avanzados?
    \vspace{3mm}

    \subsection{1.3 Objetivos}

    \textbf{Objetivo general}

    La presente tesis tiene como objetivo desarrollar un servicio de computación con capacidades de paralelización, el cual estará disponible para la comunidad universitaria y se enfocará en apoyar la investigación en el ámbito académico. Para lograr esta meta, se propone aprovechar los recursos existentes en el Departamento de Matemáticas de la Universidad del Valle. Este servicio busca optimizar el uso de los recursos informáticos y permitir una mayor eficiencia en el procesamiento de datos, mejorando así las posibilidades de investigación y contribuyendo al avance de la comunidad académica en general.
    \vspace{3mm}

    \textbf{Objetivos específicos}
    \begin{enumerate}
        \item Identificar recursos disponibles y necesidades investigativas.
        \item Diseñar una solución que considere los requerimientos de los usuarios y aproveche las capacidades de los recursos computacionales mediante la implementación de un gestor de cola de tareas.
        \item Llevar a cabo la instalación, documentación y puesta a punto de herramientas que apoyen los procesos de investigación y docencia en el área de matemáticas, facilitando el uso del servicio.
        \item  Llevar a cabo pruebas de uso de la infraestructura y de las aplicaciones desplegadas en este trabajo.
    \end{enumerate}
    \vspace{3mm}

\end{multicols}

\begin{table}[ht]
    \centering
    \caption{Productos Esperados}
    \begin{tabular}{p{7cm}p{7cm}}
      \hline
      \centering\textbf{Objetivos Específicos}                                                                                                                                                           & \textbf{Producto(s) Esperados}                                                                                                                                  \\
      \hline
      \text Identificar recursos disponibles y necesidades investigativas.                                                                                                                               & Documento expresando la arquitectura de los recursos y las necesidades a considerar.                                                                            \\
      \hline
      \text Diseñar una solución que considere los requerimientos de los usuarios y aproveche las capacidades de los recursos computacionales mediante la implementación de un gestor de cola de tareas. & Documentación, scripts, programas y pruebas básicas de operación que simplifiquen o automaticen el mantenimiento y administración del gestor de cola de tareas. \\
      \hline
      \text Llevar a cabo la instalación, documentación y puesta a punto de herramientas que apoyen los procesos de investigación y docencia en el área de matemáticas, facilitando el uso del servicio. & Documentación, scripts y programas que automaticen y faciliten la utilización del servicio para los estudiantes o profesores.                                   \\
      \hline
      \text Llevar a cabo pruebas de uso de la infraestructura y de las aplicaciones desplegadas en este trabajo.                                                                                        & Reporte de pruebas en donde se evidencie la correcta funcionalidad y la eficiencia de los recursos.                                                             \\
      \hline
    \end{tabular}
    \label{table:table1}
  \end{table}

\mylinespacing
\mylinespacing
\begin{tightcenter}
\end{tightcenter}
\end{doublespace}
