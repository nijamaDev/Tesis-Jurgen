\begin{spacing}{1.5}
  \begin{tightcenter}
    \section{7. Discusión y conclusiones}
    \mylinespacing
  \end{tightcenter}

  A lo largo de este proyecto, se han obtenido logros sustanciales en relación con la implementación de un sistema distribuido destinado a la realización de investigaciones científicas en el Departamento de Matemáticas de la Universidad del Valle. Los resultados obtenidos demuestran la viabilidad de integrar diversas herramientas y tecnologías con el propósito de establecer un sistema eficiente y escalable que facilite la realización sistemática de investigaciones. 

Este servicio de computación paralela ha demostrado ser una herramienta efectiva para mejorar el procesamiento de datos en el ámbito académico. Adicionalmente, se han identificado oportunidades de mejora que contribuirán al incremento de la eficiencia, capacidad y facilidad de uso en un futuro. Aunque se presentaron algunos problemas durante el desarrollo del proyecto, estos fueron resueltos con éxito.

Durante el proceso de pruebas, hemos obtenido resultados que nos han dado una visión clara sobre el comportamiento esperado del sistema en diversas circunstancias. Además, los datos recopilados durante las pruebas permitieron obtener una mayor comprensión de las fortalezas y debilidades del sistema, lo que permitirá mejorar su rendimiento y eficacia en el futuro de manera significativa.

Este servicio representa un importante logro en el campo de la computación científica y tiene un impacto positivo en el desarrollo de la investigación en la Universidad del Valle y en la comunidad académica en general, ya que ha brindado a los investigadores de la Universidad del Valle una plataforma para llevar a cabo sus proyectos de investigación. La capacidad de paralelización ha permitido la ejecución simultánea de múltiples tareas, lo que ha acelerado los tiempos de procesamiento y ha aumentado la productividad en la obtención de resultados.

En términos generales, este proyecto ha establecido las bases necesarias para la implementación de sistemas distribuidos en la universidad, lo cual habilitará a estudiantes y docentes a aprovechar al máximo las capacidades de la computación distribuida con el fin de llevar a cabo investigaciones de mayor complejidad y sofisticación.

Finalmente, el objetivo principal de esta tesis, que era desarrollar un servicio de computación con capacidades de paralelización para la comunidad universitaria, no solo resultó siendo un proyecto completamente viable para la universidad, sino que se ha logrado con éxito. A través del aprovechamiento de los recursos existentes en el Departamento de Matemáticas de la Universidad del Valle, se ha creado un servicio que potencia el uso de los recursos informáticos mediante la paralelización.

  \mylinespacing
  \mylinespacing
  \begin{tightcenter}
  \end{tightcenter}
\end{spacing}