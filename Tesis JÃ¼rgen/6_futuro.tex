\begin{spacing}{1.5}
  \begin{tightcenter}
    \section{6. Trabajo Futuro}
    \mylinespacing
  \end{tightcenter}

  En esta sección se presentan las mejoras que se podrían aplicar en un futuro para mejorar el sistema distribuido propuesto. Estas mejoras tienen como objetivo aumentar la eficiencia y el rendimiento del sistema, así como mejorar la accesibilidad y la capacidad de procesamiento.

  \subsection{6.1 Implementación de Cuotas de Disco}

  Una posible mejora para el sistema es la implementación de cuotas de disco para limitar el uso de espacio en disco por usuario o grupo. Esta funcionalidad permitiría una mejor gestión de los recursos de almacenamiento y evitaría que un solo usuario o grupo de usuarios consuma la capacidad de almacenamiento disponible. Esto es posible con el sistema de archivos XFS instalado.

  \subsection{6.2 Integración de Jupyterhub con Mathematica}

  Se recomienda la integración de Jupyterhub con Mathematica para permitir la ejecución de código de Mathematica desde la plataforma Jupyterhub. Esta mejora aumentaría la eficiencia y la versatilidad del sistema y permitiría a los usuarios aprovechar las capacidades de ambas herramientas.

  \subsection{6.3 Ampliación del Soporte a MATLAB}

  Se sugiere ampliar el soporte a Matlab ya que la universidad cuenta con la licencia para este software. Esta mejora aumentaría la versatilidad del sistema y permitiría a los usuarios realizar investigaciones en una variedad de campos y disciplinas.

  \subsection{6.4 Investigación de Alternativas a SLURM}

  Otra posible mejora es la investigación de alternativas a SLURM para manejar todos los recursos de una forma más centralizada. HTCondor es una opción interesante que podría ser investigada para determinar si es viable y adecuada para las necesidades de la Universidad.

  \subsection{6.5 Cursos y Talleres para Mejorar Conocimientos Técnicos}

  Se recomienda la realización de cursos y talleres para mejorar los conocimientos técnicos de los estudiantes y profesores con respecto a las herramientas implementadas y a la computación distribuida. Estos cursos y talleres permitirían a los usuarios del sistema aprovechar al máximo sus capacidades y aumentar la eficacia de sus investigaciones.

  \subsection{6.6 Ampliación del Soporte a Inteligencia Artificial}

  Se sugiere la utilización del clúster para entrenar modelos en Tensor Flow y amplificar la investigación en Inteligencia Artificial. Esta mejora permitiría a los usuarios del sistema realizar investigaciones más complejas y sofisticadas en el campo de la Inteligencia Artificial.

  \subsection{6.7 Implementación de Ansible}

  Se recomienda la implementación de Ansible como herramienta de automatización para facilitar la configuración y la administración del sistema distribuido. Esta mejora permitiría una gestión más eficiente del sistema y reduciría la carga de trabajo administrativo.

  \subsection{6.8 Mejoras en la Seguridad del Sistema}

  Se sugiere mejorar la seguridad del sistema mediante la configuración de acceso de ssh solo con llave y no con contraseña. Esta medida aumentaría la seguridad del sistema y protegería la información y los recursos almacenados.

  \subsection{6.9 Conseguir Financiación para la Adquisición de Nuevos Nodos}

  Se recomienda conseguir financiación para comprar más nodos para el clúster, modernizando y estabilizando los sistemas. La adquisición de nuevos nodos permitiría aumentar la capacidad de procesamiento y la eficiencia del sistema.
  \begin{tightcenter}
  \end{tightcenter}
\end{spacing}