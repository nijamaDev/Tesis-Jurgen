\begin{doublespace}
\begin{tightcenter}
\section{3. Implementación y pruebas}
\mylinespacing
\end{tightcenter}

\begin{multicols}{2}
    \subsection{3.1 Implementando tecnologías}

\textbf{Procedimiento}

\subsection{3.2 Problemas encontrados}
En el desarrollo de este proyecto, se presentaron problemas tanto previstos como inesperados, los cuales serán mencionados a continuación.

\subsubsection{3.2.1 Problemas esperados}

\begin{enumerate}
    \item \textbf{Heterogeneidad de los recursos computacionales:} El clúster Bochica y la sala de computación Jürgen Tischer contienen una variedad de recursos computacionales, incluyendo diferentes tipos de computadoras, sistemas operativos y versiones de software. Esto puede dificultar la optimización del sistema distribuido diseñado y requerir una mayor planificación y flexibilidad en el diseño y la implementación.
    \item \textbf{Antigüedad de los computadores:} Algunos de los computadores en el clúster Bochica y la sala de computación Jürgen Tischer son antiguos y pueden tener un impacto en la eficiencia y la capacidad de ejecutar tareas de investigación de manera óptima.
    \item \textbf{Dificultades en la adaptación a las nuevas herramientas:} La implementación de nuevas herramientas y tecnologías puede requerir un período de adaptación y aprendizaje para los usuarios, lo que puede retrasar los procesos de investigación y enseñanza. Además, pueden surgir problemas técnicos durante la instalación y configuración de las herramientas, lo que puede interferir en la eficiencia y productividad.
\end{enumerate}

    \subsubsection{3.2.2 Problemas no esperados}

    \textbf{Problemas con el software}

    \begin{itemize}
        \item Problemas de licencias: Parte del software utilizado era propietario y resultó problematico el correcto uso de estas licencias, especialmente el software de Wolfram Mathematica.
    \end{itemize}


    \textbf{Problemas con el Hardward}

    \begin{itemize}
        \item Cables mal acomodados: Los cables del clúster estaban mal acomodados, lo que generaba una dificultad para conocer las diferentes interconexiones entre los recursos.
        \item Cables faltantes: Se encontró que se requería un cable serial para la adecuada configuración de un switch que permite la interconexión entre los equipos del clúster.
        \item Permisos olvidados: Varios equipos del clúster debido a su desuso, se habían perdido las credenciales para utilizarlos de la manera adecuada
        \item Partes que requerían mantenimiento: Algunas partes del hardware requerían mantenimiento para su óptimo funcionamiento, pero esto no se había realizado.
        \item Desconocimiento de las limitaciones del hardware: Al principio no se conocían las limitaciones del hardware, lo que dificultaba su correcto uso y aprovechamiento.
        \item Hardware mal acomodado: El hardware estaba mal acomodado, lo que generaba problemas en la conexión y en el acceso a los recursos computacionales.
    \end{itemize}

    \textbf{Problemas con la Implementación del Sistema Distribuido}

    \begin{itemize}
        \item Falta de documentación clara para la implementación: La documentación para la implementación del sistema distribuido no era clara, lo que dificultaba su correcta implementación.
        \item Dificultades en la integración de los diferentes componentes del sistema: Se encontraron dificultades en la integración de los diferentes componentes del sistema distribuido, lo que limitaba su correcto funcionamiento.
        \item Limitaciones en la capacidad de paralelización: Se encontraron limitaciones en la capacidad de paralelización del sistema distribuido, lo que disminuía su eficiencia y efectividad.\newline
    \end{itemize}

    \textbf{Problemas con las Herramientas Instaladas}

    \begin{itemize}
        \item Falta de compatibilidad con otras aplicaciones: Las herramientas instaladas no eran compatibles con otras aplicaciones, lo que limitaba su uso y efectividad.
        \item Dificultades en la configuración y uso: Se encontraron dificultades en la configuración y uso de las herramientas instaladas, lo que disminuía su efectividad.
        \item Falta de documentación y apoyo técnico: La falta de documentación y apoyo técnico para las herramientas instaladas limitaba su uso y efectividad.
    \end{itemize}

    \textbf{Problemas con las Pruebas de Rendimiento}

    \begin{itemize}
        \item Falta de recursos y tiempo para realizar las pruebas: No se contaba con los recursos y tiempo necesario para realizar las pruebas de rendimiento, lo que limitaba la evaluación de la infraestructura y las aplicaciones implementadas.
        \item Falta de una metodología clara para la realización de las pruebas: No había una metodología clara para la realización de las pruebas de rendimiento, lo que generaba incertidumbre en los resultados y dificultades en la interpretación de los mismos.
        \item Dificultades en la comparación de resultados con otras infraestructuras: Se encontraron dificultades en la comparación de los resultados obtenidos con otras infraestructuras similares, lo que disminuía la validez de los resultados.
        \item Falta de un sistema de seguimiento y monitoreo de las pruebas: No había un sistema de seguimiento y monitoreo de las pruebas, lo que dificultaba la identificación y solución de posibles problemas y limitaba la mejora continua de la infraestructura.
    \end{itemize}

    \subsection{3.3 Pruebas de control}

    \subsection{3.4 Pruebas de eficiencia}

    \subsection{3.5 Automatización - Facilitar el uso}

    \subsubsection{3.5.1 Scripts}

    En el desarrollo de este proyecto, se llevaron a cabo diversos scripts de automatización que permiten realizar tareas de mantenimiento y configuración de manera más eficiente y sistemática. Estos scripts han sido diseñados para optimizar procesos específicos dentro del proyecto y su implementación ha permitido reducir el tiempo de ejecución de tareas repetitivas y mejorar la calidad del trabajo.

    Estas son las funciones principales de los scripts de automatización:

    \begin{itemize}
        \item Encendido y apagado de computadores.
        \item Verificación del estado de los computadores y servicios.
        \item Actualización de software.
        \item Instalación de paquetes de software.
        \item Configuración completa de un nuevo equipo recién instalado o de un equipo antiguo formateado.
    \end{itemize}
   
    Esto permitirá minimizar las tareas comunes y de mantenimiento requeridas con menor esfuerzo. Además, estos scripts son altamente escalables y pueden ser adaptados para su uso en proyectos futuros, lo que representa una inversión a largo plazo en la mejora de la eficiencia y la calidad del trabajo.

\end{multicols}

\mylinespacing
\mylinespacing
\begin{tightcenter}
\end{tightcenter}
\end{doublespace}
