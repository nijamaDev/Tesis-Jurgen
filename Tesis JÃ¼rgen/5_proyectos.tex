\begin{spacing}{1.5}
  \begin{tightcenter}
    \section{5. Influencia en Proyectos Destacados}
  \end{tightcenter}

  En este apartado queremos exponer algunos trabajos que se han visto
  beneficiados por el producto de esta tesis.

  \begin{enumerate}

    \item \textbf{Tesis de pregrado:} Dinámica del universo con Campos
          Taquiónicos. \newline
          \textbf{Datos de los investigadores:}
          \begin{itemize}
            \item Santiago Garcia Serna, estudiante de pregrado en Física
            \item Cesar Alonso Valenzuela Toledo, Ph.D. Profesor del
                  departamento de Física.
            \item Hernan Ocampo Duran, Ph.D. Profesor del departamento de
                  Física
          \end{itemize}
          \textbf{Departamento: } Física, Universidad del Valle \newline
          \textbf{Publicaciones asociadas al proyecto: } \newline Reconstructing
          the parameter space of non-analytical cosmological fixed points.\cite{Tesis1}

    \item \textbf{Nombre del proyecto:} Construcción numérica de datos
          iniciales en relatividad general. \newline
          \textbf{Datos de los investigadores:}
          \begin{itemize}
            \item Alejandro Estrada Llesta, estudiante de maestría en
                  matemáticas.
            \item Leon Escobar Diaz, Ph.D. Profesor del departamento de
                  Matemáticas.
          \end{itemize}
          \textbf{Departamento: } Matemáticas, Universidad del Valle. \newline
          \textbf{Publicaciones asociadas al proyecto: } \newline An hyperbolic
          approach for numerical constructing initial data sets of cosmological
          spacetimes. (En preparación)

    \item \textbf{Nombre del proyecto:} Exploración numérica de la ecuación
          de Benjamin-Ono en mallas no uniformes y con condiciones de frontera. \newline
          \textbf{Datos de los investigadores:} \newline
          Leon Escobar Diaz, Ph.D. Profesor del departamento de Matemáticas.
          \newline
          \textbf{Departamento: } Matemáticas, Universidad del Valle. \newline
          \textbf{Publicaciones asociadas al proyecto: } \begin{enumerate}
            \item A spectral-infinite element method for solving the hyperbolic
                  Einstein constraint equations. (En revisión)
            \item Numerical construction of asymptotically flat static
                  spacetimes from a Bartnik data set. (En progreso)
          \end{enumerate}

    \item \textbf{Nombre del proyecto:} Representación de enteros en
          sucesiones generalizadas de fibonacci y coordenadas de la ecuación de Pell.
          \newline
          \textbf{Datos de los investigadores:} \newline
          Carlos Alexis Gómez Ruiz, Ph.D. Profesor del departamento de
          Matemáticas. \newline
          \textbf{Departamento: } Matemáticas, Universidad del Valle. \newline
          \textbf{Publicaciones asociadas al proyecto: } \newline Por establecer.

    \item \textbf{Nombre del proyecto:} Aspectos numéricos y aplicaciones
          de matrices Toeplitz. \newline
          \textbf{Datos de los investigadores:} \newline
          Manuel Bogoya, Ph.D. Profesor del departamento de Matemáticas.\newline
          \textbf{Departamento: } Matemáticas, Universidad del Valle. \newline
          \textbf{Publicaciones asociadas al proyecto: } \newline Fast Toeplitz
          eigenvalue computations, joining interpolation-extrapolation matrix-less
          algorithms and simple-loop theory. \cite{Proy4}

    \item \textbf{Nombre del proyecto:} Simulación de micro-nadadores a
          través de métodos de elementos finitos acoplados a técnicas de aprendizaje de
          máquina. \newline
          \textbf{Datos de los investigadores:} \newline
          Stevens Paz, Ph.D. Profesor del departamento de Matemáticas. \newline
          \textbf{Departamento: } Matemáticas, Universidad del Valle. \newline
          \textbf{Publicaciones asociadas al proyecto: } \newline Chemoreception
          and chemotaxis of a three-sphere swimmer, (en revisión)\newpage

    \item \textbf{Nombre del proyecto:} Una variante del problema de Pillai en sucesiones recurrentes lineales. \newline
          \textbf{Datos de los investigadores:} \newline
          Carlos Alexis Gómez Ruiz, Ph.D. Profesor del departamento de Matemáticas. \newline
          \textbf{Departamento: } Matemáticas, Universidad del Valle. \newline
          \textbf{Publicaciones asociadas al proyecto: } \newline CI 71280 (pendiente por confirmación).

    \item \textbf{Nombre del proyecto:} Representacion de enteros en sucesiones generalizadas de Fibonacci y coordenadas de la ecuacion de Pell. \newline
          \textbf{Datos de los investigadores:} \newline
          Carlos Alexis Gómez Ruiz, Ph.D. Profesor del departamento de Matemáticas. \newline
          \textbf{Departamento: } Matemáticas, Universidad del Valle. \newline
          \textbf{Publicaciones asociadas al proyecto: } \newline CI 71327 (pendiente por confirmación).

  \end{enumerate}

  \mylinespacing
  \mylinespacing
  \begin{tightcenter}
  \end{tightcenter}
\end{spacing}