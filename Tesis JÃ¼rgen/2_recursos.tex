\begin{spacing}{1.5}
    \begin{tightcenter}
      \section{2. Análisis de recursos y necesidades}
      \mylinespacing
    \end{tightcenter}

  En este capítulo se hace un análisis detallado de los recursos informáticos disponibles para esta tesis en el Departamento de Matemáticas de la Universidad del Valle.

  \subsubsection{2.1 Recursos computacionales}
  El departamento de Matemáticas cuenta con dos espacios dedicados a la utilización de recursos computacionales: un clúster computacional y una sala de cómputo.

  En la siguiente tabla se muestran los computadores y servidores con los que cuentan dichos espacios:
  \vspace{3mm}

  \begin{table}[h!]
    \centering
    \begin{tabular}{p{5cm}|c|c|c}
    \hline
    \textbf{Rubro} & \textbf{Núcleos por Unidad} & \textbf{Unidades} & \textbf{Total Núcleos} \\ \hline
    \textbf{Servidores en Clúster Bochica} &  &  &  \\ \hline
    COMPUTADOR HP DL360E GEN 8 & 16 & 4 & 64 \\ \hline
    SERVIDOR DELL POWEREDGE 1950 & 8 & 4 & 32 \\ \hline
    \textbf{Subtotal} &  & \textbf{8} & \textbf{96} \\ \hline
    \textbf{Computadores en Sala de Cómputo Jürgen Tischer} &  &  &  \\ \hline
    COMPUTADOR WORKSTATION HP Z1 & 10 & 2 & 20 \\ \hline
    COMPUTADOR WORKSTATION HP Z2 & 8 & 1 & 8 \\ \hline
    COMPUTADOR DELL PRECISION T3610 & 4 & 30 & 120 \\ \hline
    COMPUTADOR DELL PRECISION T5500 & 2 & 7 & 14 \\ \hline
    \textbf{Subtotal} &  & \textbf{40} & \textbf{162} \\ \hline
    \textbf{Total} & \textbf{ } & \textbf{48} & \textbf{258} \\ \hline
    \end{tabular}
    \caption{Recursos Computacionales}
    \label{table:table1}
    \end{table}

    Adicionalmente, los equipos cuentan con unidades de alimentación ininterrumpida (UPS, por sus siglas en inglés), a la par de un dispositivo de conmutación de red (switch) que interconecta a los servidores presentes en el Clúster. Es importante destacar que todos los equipos informáticos mantienen una conexión con la red universitaria perteneciente al campus y, asimismo, disponen de acceso a internet.

    \subsubsection{2.2 Recursos de software}

    La Universidad del Valle dispone de licencias de software que pueden resultar de gran utilidad para el desarrollo de esta tesis, entre ellas se encuentran Wolfram Mathematica, Mathematica Licence Manager, gridMathematica, MATLAB, MATLAB Licence Manager y MATLAB Parallel Server. Además de esto se utilizarán licencias de software de uso libre.

    \subsubsection{2.3 Necesidades por parte del departamento de Matemáticas}

    El profesor León Escobar, quien está a cargo de la sala de Matemáticas Jürgen Tischer y es co-director de esta tesis, ha destacado ciertas características que deben ser tomadas en cuenta en el desarrollo de esta tesis. Entre los principales aspectos destacados se encuentra la oportunidad de emplear computación paralela, utilizando diferentes lenguajes y herramientas como Python, C++, R, Wolfram Mathematica y MATLAB.
    Además, el profesor ha mencionado que prefiere el uso de Slurm como gestor de cola de tareas, ya que es ampliamente conocido en el ámbito académico y familiar para él.

  
    \mylinespacing
    \mylinespacing
    \begin{tightcenter}
    \end{tightcenter}
  \end{spacing}
  