\begin{doublespace}
\begin{tightcenter}
\section{2. Diseño y metodología}
\mylinespacing
\end{tightcenter}

\begin{multicols}{2}

    En este capítulo se describen los métodos utilizados para el diseño y desarrollo del servicio de computación, así como el plan para su implementación.

    \subsection{Resultados esperados}
\end{multicols}

\begin{table}[ht]
    \centering
    \begin{tabular}{p{7cm}p{7cm}}
      \hline
      \centering\textbf{Objetivos Específicos}                                                                                                                                                           & \textbf{Producto(s) Esperados}                                                                                                                                  \\
      \hline
      \text Identificar recursos disponibles y necesidades investigativas.                                                                                                                               & Documento expresando la arquitectura de los recursos y las necesidades a considerar.                                                                            \\
      \hline
      \text Diseñar una solución que considere los requerimientos de los usuarios y aproveche las capacidades de los recursos computacionales mediante la implementación de un gestor de cola de tareas. & Documentación, scripts, programas y pruebas básicas de operación que simplifiquen o automaticen el mantenimiento y administración del gestor de cola de tareas. \\
      \hline
      \text Llevar a cabo la instalación, documentación y puesta a punto de herramientas que apoyen los procesos de investigación y docencia en el área de matemáticas, facilitando el uso del servicio. & Documentación, scripts y programas que automaticen y faciliten la utilización del servicio para los estudiantes o profesores.                                   \\
      \hline
      \text Llevar a cabo pruebas de uso de la infraestructura y de las aplicaciones desplegadas en este trabajo.                                                                                        & Reporte de pruebas en donde se evidencie la correcta funcionalidad y la eficiencia de los recursos.                                                             \\
      \hline
    \end{tabular}
    \caption{Productos Esperados}
    \label{table:table3}
  \end{table}

  \begin{table}[ht]
    \centering
    \begin{tabular}{m{4.6cm}m{4.6cm}m{4.6cm}}
      \hline
      \centering\textbf{Objetivos Específicos} & \textbf{Actividad} & \textbf{Resultado Esperado por Actividad}                                                                                                                                 \\
      \hline
      \text Identificar recursos disponibles y necesidades investigativas. & Identificar la totalidad de los recursos disponibles y las necesidades investigativas para el proyecto. & Documento expresando la arquitectura de los recursos y las necesidades a considerar. \\
      \hline
      \multirow{2}{4.3cm}{Llevar a cabo la instalación, documentación y puesta a punto de herramientas que apoyen los procesos de investigación y docencia en el área de matemáticas, facilitando el uso del servicio.} & Identificación e implementación del gestor de cola de tareas adecuado para la arquitectura y las necesidades. \\ \cline{2-2}
      & Integración de herramientas de mantenimiento y administración o creación de nuevas soluciones en caso de ser necesario. & Documentación, scripts, programas y pruebas básicas de operación que simplifiquen o automaticen el mantenimiento y administración del gestor de cola de tareas. \\
      \hline     
    \end{tabular}
    \caption{Resultados Esperados}
    \label{table:table4}
  \end{table}

\mylinespacing
\mylinespacing
\begin{tightcenter}
\end{tightcenter}
\end{doublespace}
