\begin{doublespace}
  \begin{tightcenter}
    \section{2. Diseño y metodología}
    \mylinespacing
  \end{tightcenter}

    En este capítulo se describen los métodos utilizados para el diseño y
    desarrollo del servicio de computación, así como el plan para su
    implementación.

    \subsection{2.1 Resultados esperados}

  \begin{longtable}{m{4.6cm}m{4.6cm}m{4.6cm}}
    \hline
    \textbf{Objetivos Específicos} & \textbf{Actividad} &\textbf{Resultado Esperado} \\
    \hline 
    \endfirsthead
    % aquí añadimos el encabezado del resto de hojas.
    \hline
    \textbf{Objetivos Específicos} & \textbf{Actividad} &\textbf{Resultado Esperado } \\
    \hline 
    \endhead
    % aquí añadimos el fondo de todas las hojas, excepto de la última.
    \multicolumn{2}{c}{Sigue en la página siguiente.}
    \endfoot
    \endlastfoot
    \text Identificar recursos disponibles y necesidades investigativas. &
    Identificar la totalidad de los recursos disponibles y las necesidades
    investigativas para el proyecto.                                     &
    Documento expresando la arquitectura de los
    recursos y las necesidades a considerar.
    \\
    \hline    
    \multirow{2}{4.3cm}{Llevar a cabo la instalación, documentación y puesta
      a punto de herramientas que apoyen los procesos de investigación y
      docencia en
    el área de matemáticas, facilitando el uso del servicio.}            &
    Identificación e
    implementación del gestor de cola de tareas adecuado para la arquitectura
    y las
    necesidades.                                                         &
    Documentación, scripts, programas y pruebas básicas de operación
    que simplifiquen o automaticen el mantenimiento y administración del
    gestor de
    cola de tareas.
    \\ \cline{2-3}
                                                                         &
    Integración de herramientas de mantenimiento y administración o
    creación de nuevas soluciones en caso de ser necesario.              &
    Documentación,
    scripts, programas y pruebas básicas de operación que simplifiquen o
    automaticen el mantenimiento y administración del gestor de cola de
    tareas.                                                                \\
    \hline
    \multirow{2}{4.3cm}{Llevar a cabo la instalación, documentación y puesta a punto de herramientas que apoyen los procesos de investigación y docencia en el área de matemáticas, facilitando el uso del servicio.}            &
    Identificación de puntos de automatización.                                                         &
    Documentación, scripts y programas que automaticen y faciliten la utilización del servicio para los estudiantes o profesores.
    \\ \cline{2-3}
                                                                         &
    Integración de herramientas de apoyo según las necesidades identificadas o creación de nuevas soluciones en caso de ser necesario.            &
    Documentación, scripts y programas que automaticen y faciliten la utilización del servicio para los estudiantes o profesores.                                                               \\
    \hline
    \multirow{2}{4.3cm}{Llevar a cabo pruebas de uso de la infraestructura y de las aplicaciones desplegadas en este trabajo.}            &
    Diseñar y realizar pruebas de correcta funcionalidad.                                                         &
    Reporte de pruebas en donde se evidencie la correcta funcionalidad del servicio y arquitectura.
    \\ \cline{2-3}
                                                                         &
    Diseñar y realizar pruebas de eficiencia.            &
    Reporte de pruebas en donde se evidencie la eficiencia del sistema.                                                          \\
    \hline
    \\ 
    \caption{Resultados esperados}
    \label{table:table3}
    \end{longtable}

    \subsection{2.2 Escogiendo tecnologías a utilizar}
    En esta sección se presentan las diversas especificaciones tecnológicas seleccionadas para el proyecto en cuestión.

    \subsubsection{2.2.1 Sistema operativo}
    El sistema operativo utilizado para la instalación de los equipos en este proyecto es Rocky Linux.

    \textbf{¿Qué es Rock Linux?}
    
    Rocky Linux es una distribución de Linux que se basa en el código fuente de Red Hat Enterprise Linux (RHEL), el cual se compila y se distribuye de forma gratuita y de código abierto. Rocky Linux está diseñado para ser compatible con el mismo software, herramientas y aplicaciones que RHEL, y ofrece una alternativa de código abierto para los usuarios que anteriormente utilizaban CentOS. Lo cual resulta en una herramienta poderosa por la gran cantidad de software disponible para los entornos RHEL.

    Este tipo de distribución tipo enterprise está diseñada para ofrecer estabilidad y confiabilidad en entornos de producción. La distribución incluye una variedad de herramientas y utilidades que facilitan la gestión y el mantenimiento de sistemas informáticos, incluyendo servidores y estaciones de trabajo. Además, Rocky Linux es compatible con paquetes de software de terceros y puede ejecutar aplicaciones de Linux de forma nativa. \cite{RL-1}

    \textbf{Rocky Linux 9}
    
    La versión específica del software utilizado es Rocky Linux 9 (*complete iso o DVD iso*) \cite{RL9-download-1} \cite{RL9-release-1}  \cite{RHEL-release-1} 

    Rocky Linux 9 ofrece soporte para actualizaciones de seguridad de este software hasta el 31 de Mayo de 2032 \cite{RL9-EOL-1}. Debido su estabilidad y largo ciclo de vida que ofrece, es ideal para minimizar las actualizaciones manuales a versiones mayores, lo que simplifica el mantenimiento requerido en el largo plazo.

    \textbf{Kernel}

    El kernel o núcleo es la parte central de un sistema operativo. Es el componente que se encarga de controlar el hardware y los recursos del sistema. El kernel es responsable de administrar la memoria del sistema, asignando y liberando recursos de manera eficiente. También se encarga de administrar los procesos y subprocesos del sistema, lo que permite que múltiples programas se ejecuten al mismo tiempo.

    El kernel es esencial para el funcionamiento del sistema operativo y proporciona una capa de abstracción entre el hardware y el software.  \cite{RHEL-kernel-1}

    El kernel incluido en Rocky Linux 9 es la versión 5.14 LTS, que es bastante reciente. Esto significa que tanto los ordenadores antiguos, así como los más nuevos en la sala de cómputo pueden beneficiarse de esta actualización, y aquellos que se planeen comprar en el futuro cercano también podrán aprovecharla. \cite{RL9-release-1}

    \subsubsection{2.2.2 Sistema de Archivos}

    \textbf{XFS}
    
    Rocky Linux, al igual que RHEL utiliza por defecto XFS como sistema de archivos. XFS es un sistema de archivos de alto rendimiento que se utiliza en sistemas operativos Linux. Fue desarrollado por Silicon Graphics en la década de 1990 y se ha incorporado en el kernel de Linux desde la versión 2.4.

    XFS está diseñado para manejar grandes volúmenes de datos y archivos. Algunas de las características de XFS incluyen:

    \begin{itemize}
        \item Alta capacidad de lectura y escritura de archivos.
        \item Sistema de administración de archivos de registro que permite una recuperación rápida después de un fallo del sistema.
        \item Sistema de cuotas de disco para limitar el uso de espacio en disco por usuario o grupo.
    \end{itemize}

    XFS es una buena opción para sistemas que manejan grandes cantidades de datos y aplicaciones que requieren una alta capacidad de lectura y escritura de archivos. \cite{RHEL-XFS-1}

    \textbf{NFS}
    
    NFS (Network File System) es un protocolo de red de alto rendimiento y eficiente que permite a los sistemas informáticos compartir archivos y recursos a través de una red. Está diseñado para reducir la carga en la red y en el servidor. El protocolo utiliza una caché para reducir la cantidad de solicitudes de red y aumentar la velocidad de acceso a los recursos compartidos. Además, soporta la transferencia de datos en modo ráfaga, lo que permite que los datos se transfieran en grandes bloques en lugar de pequeños paquetes, lo que aumenta la eficiencia de la transferencia de datos. Es un estándar para compartir archivos en sistemas operativos basados en Unix y Linux.

    NFS se basa en el modelo cliente-servidor, donde un servidor exporta un sistema de archivos o un directorio a través de la red, y los clientes pueden montar este sistema de directorio en sus propios sistemas, lo que les permite acceder a los archivos y recursos compartidos como si estuvieran en su propia máquina.

    El protocolo NFS utiliza un sistema de autenticación y autorización para controlar el acceso a los recursos compartidos. Los clientes deben autenticarse con el servidor antes de acceder a los recursos compartidos, y el servidor puede configurar los permisos de acceso en base a las credenciales de los usuarios. \cite{RHEL-NFS-1} \cite{RHEL-NFS-2}

    \subsubsection{2.2.3 Distribución de Aplicaciones}
    Existen diversas opciones para la instalación de aplicaciones en Rocky Linux. Una de las formas más comunes es a través del uso de gestores de paquetes, como DNF, que permiten descargar e instalar software desde repositorios compatibles. Además, también es posible utilizar Flatpak, que ofrece una mayor flexibilidad en cuanto a la selección de versiones de aplicaciones y resolución de dependencias. Otra opción es mediante el uso de Podman, una alternativa más segura a Docker, aunque esta alternativa puede resultar más compleja y no es recomendada para usuarios inexpertos.

    En nuestro caso se planificó el uso de Flatpak con Flathub como medio principal para la instalación de aplicaciones, debido a que se ejecuta en un ambiente encapsulado, evitando así conflictos con las dependencias del sistema. Las aplicaciones o librerías que no se pueden obtener de Flathub se instalan mediante distintos repositorios.

    \textbf{Repositorios}
    
    Rocky Linux de base solo cuenta con sus propios repositorios de software, lo que acarrea un problema en la insuficiente variedad de programas y herramientas. Sin embargo, también existen otros repositorios compatibles populares como **EPEL** (Extra Packages for Enterprise Linux), que proporciona software adicional para sistemas basados en Red Hat Enterprise Linux, y **RPM Fusion**, que ofrece una gran cantidad de paquetes de software de código abierto que no están disponibles en los repositorios oficiales de la distribución. Al tener acceso a estos repositorios, se puede disfrutar de una selección de software y herramientas más completa. \cite{RL-repo-1} \cite{RHEL-EPEL-1} \cite{rpmfusion-1}

    \textbf{Flatpak}
    
    Flatpak es un sistema de paquetes de software que permite la distribución de aplicaciones de forma independiente de la distribución de Linux. Esto significa que los usuarios de Linux pueden descargar e instalar aplicaciones de una tienda centralizada (Flathub) sin preocuparse por las diferencias en la distribución de Linux que estén utilizando.

    Además, Flatpak se basa en contenedores, lo que significa que las aplicaciones y todas sus dependencias se empaquetan en un contenedor aislado. Esto permite a las aplicaciones ejecutarse de forma independiente de otras aplicaciones y bibliotecas del sistema, lo que ayuda a garantizar la estabilidad y la seguridad.

    También existe una tienda centralizada donde los usuarios pueden descargar e instalar aplicaciones. La tienda se llama Flathub, es un lugar con una amplia selección de software disponible sin costo alguno.

    La mayor ventaja de Flatpak y la razón por la que se escoge como método principal para instalar aplicaciones en el sistema es debido al aislamiento de dependencias, lo que evita conflictos con el sistema operativo. \cite{FLAT-1} \cite{FLAT-2} \cite{RHEL-FLAT-1} \cite{PHOENIX-FLAT-1}

    \textbf{Podman}
    
    Aunque se proveen las librerías y aplicaciones más comunes en la configuración que se realizó, con los distintos repositorios mencionados y el uso de flatpaks, hay usuarios que podrían tener aún más necesidades y más específicas, por lo que también se tiene Podman como soporte para usuarios avanzados.

    Podman es una herramienta de administración de contenedores para Linux. Permite a los usuarios crear, ejecutar y gestionar contenedores de manera similar a otras herramientas populares, como Docker. Sin embargo, a diferencia de Docker, Podman no requiere un servicio del sistema para ejecutar los contenedores, lo que significa que estos se ejecutan como procesos normales del usuario y pueden administrarse utilizando herramientas y comandos. Además, Podman utiliza el concepto de "pods" en lugar de "servicios", lo que permite a los usuarios agrupar y gestionar varios contenedores relacionados juntos en una única entidad lógica. 
    
    Podman es una herramienta de código abierto y se desarrolla como parte del proyecto de código abierto de la comunidad de Red Hat. \cite{RHEL-podman-1}

    \subsubsection{2.2.4 Wolfram Mathematica}

    En el marco de este proyecto, se llevará a cabo una utilización efectiva de las licencias que actualmente se encuentran disponibles en la universidad. Se han identificado diversas herramientas y software matemáticos que son de gran utilidad para la realización de diversas tareas en el ámbito académico y científico, y la universidad cuenta con licencias para tres de ellas: Wolfram Mathematica, gridMathematica y MathLM.

    Estas licencias representan una valiosa oportunidad para la comunidad universitaria de utilizar herramientas de alta calidad y prestigio en el campo de las matemáticas, lo que permitirá mejorar el desarrollo de investigaciones, proyectos y actividades en general.
    
    Wolfram Mathematica es un sistema de software de álgebra computacional utilizado en matemáticas, física, ingeniería, ciencias sociales y otros campos. Ofrece una amplia gama de capacidades que incluyen análisis simbólico y numérico, visualización de datos, programación, modelado de sistemas y más. \cite{Wolfram-mathematica-1}
    
    MathLM es una herramienta de licencia de red de Mathematica que permite a las instituciones controlar y administrar el acceso de los usuarios al software. Con MathLM, las organizaciones pueden asignar licencias a usuarios individuales o grupos de usuarios, lo que les permite acceder al software desde cualquier lugar de la red. \cite{Wolfram-mathlm-1}
    
    gridMathematica es una herramienta de computación distribuida que permite a las instituciones y organizaciones compartir la carga de procesamiento de Mathematica en múltiples computadoras. Al distribuir el procesamiento de cálculos complejos en múltiples sistemas, gridMathematica puede reducir significativamente el tiempo necesario para realizar análisis complejos y simulaciones. También proporciona un mecanismo para administrar recursos computacionales y asegurar que las tareas se distribuyan de manera eficiente. Wolfram gridMathematica es una extensión de las capacidades de paralelización integradas de Mathematica que ejecuta más tareas en paralelo, en más CPUs y GPUs, para una ejecución más rápida. Automatiza la coordinación y gestión de procesos sin necesidad de cambios en el código. \cite{Wolfram-grid-1}

    \subsubsection{2.2.5 SLURM}

    Como se ha mencionado anteriormente SLURM (acrónimo de Simple Linux Utility for Resource Management) es un sistema de gestión de recursos utilizado en entornos de computación de alto rendimiento (HPC). Algunas de las razones por las que se escoge Slurm para la gestión de recursos son:

    \begin{enumerate}
      \item Escalabilidad: Slurm es altamente escalable y puede manejar eficientemente grandes cargas de trabajo y múltiples usuarios en entornos de computación de alto rendimiento.
      \item Flexibilidad: Slurm es altamente configurable y permite a los usuarios definir y personalizar políticas de gestión de recursos, tales como prioridades de trabajos, particiones de clusters, políticas de colas, etc. Esto permite que Slurm se adapte a las necesidades específicas de cada grupo.
      \item Eficiencia: Slurm está diseñado para minimizar la sobrecarga del sistema y maximizar la eficiencia del uso de los recursos. Esto se logra mediante el uso de técnicas avanzadas de planificación y asignación de recursos, lo que garantiza que los trabajos se ejecuten de la manera más eficiente posible.
      \item Soporte de múltiples arquitecturas: Slurm soporta una amplia variedad de arquitecturas de hardware, incluyendo x86, ARM, IBM Power y otras arquitecturas de procesador.
      \item Comunidad activa: Slurm es un proyecto de código abierto con una comunidad de usuarios y desarrolladores activa y comprometida. Esto significa que hay una gran cantidad de recursos y soporte disponible en línea para ayudar a los usuarios de Slurm.
    \end{enumerate}

    Se anticipa que la implementación y utilización de Slurm resultará ser menos compleja que la de HTCondor, lo que confiere una mayor accesibilidad a la comunidad universitaria. \cite{RHEL-SLURM-1}

    \subsubsection{2.2.6 Jupyterhub}

    En el marco de este proyecto, se ha contemplado la utilización de JupyterHub como plataforma de acceso a los recursos computacionales, lo que permitirá a los usuarios disfrutar de una interfaz sencilla, centralizada y automatizada.

    JupyterHub es una plataforma sumamente versátil que se presta de manera excepcional para la gestión y ejecución simultánea de diversas instancias de Jupyter Notebook. Esta herramienta, por tanto, representa una alternativa idónea para contextos educativos o de investigación en los que varios usuarios necesitan emplear Jupyter Notebook de forma paralela.

    Al ejecutarse a través de un servidor central, el software de JupyterHub permite a los usuarios iniciar y detener sus sesiones de Jupyter Notebook de manera independiente. De esta manera, cada usuario dispone de su propio espacio de trabajo seguro y aislado en el que puede generar, llevar a cabo y compartir sus cuadernos de Jupyter, al tiempo que ofrece a los administradores de sistema un control riguroso del acceso y la seguridad de las múltiples instancias que se ejecutan.

  \mylinespacing
  \mylinespacing
  \begin{tightcenter}
  \end{tightcenter}
\end{doublespace}