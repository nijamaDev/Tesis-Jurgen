\ProvidesFile{ap-numbers.tex}[2022-07-09 numbers, units, and etc. appendix]

%  Primary sources:
%      See https://www.bipm.org/utils/common/pdf/si-brochure/SI-Brochure-9-EN.pdf.
%      See https://www.nist.gov/pml/special-publication-811/nist-guide-si-chapter-4-two-classes-si-units-and-si-prefixes.
%
%  Notes:
%      See https://www.iso.org/standard/60241.html.

\begin{VerbatimOut}{z.out}
\chapter{NUMBERS, UNITS, AND ETC.}

The \siunitxLogo\ \LaTeX\ package
\cite{wright2022}
can be used
to correctly typeset
numbers,
units,
combined units,
quantities (numbers with units or combined units),
etc.
\end{VerbatimOut}

\MyIO


%xx Note to self: scientific prefixes, scientific suffixes, tables.
%xx
%xx \begin{VerbatimOut}{z.out}
%xx
%xx \section{Number Examples}
%xx \end{VerbatimOut}
%xx
%xx \MyIO
%xx
%xx \begin{VerbatimOut}{z.out}
%xx
%xx \section{Unit Examples}
%xx \end{VerbatimOut}
%xx
%xx \MyIO
%xx
%xx \noindent\begin{tabular}{@{}lll@{}}
%xx   \textbf{Input}& \textbf{Output}& \textbf{Comment}\\
%xx   \tabularspace
%xx   \verb+\si{\kg}+& \si{\kg}& kilogram\\
%xx   \verb+\si{\m}+& \si{\m}& meter\\
%xx   \verb+\si{\kg\per\m\squared}+&
%xx     \si{\kg\per\m\squared}&
%xx     \(= \si{\kg}/\si{\m\squared}\)\\
%xx \end{tabular}
%xx \end{VerbatimOut}
%xx
%xx \MyIO
%xx
%xx \begin{VerbatimOut}{z.out}
%xx
%xx \section{Combined Number and Unit Examples}
%xx \end{VerbatimOut}
%xx
%xx \MyIO
%xx
%xx \begin{VerbatimOut}{z.out}
%xx \begin{tabular}{@{}lll@{}}
%xx   \textbf{Input}& \textbf{Output}& \textbf{Comment}\\
%xx   \tabularspace
%xx   \verb+\SI{12}{\kg}+& \SI{12}{\kg}& 12 kilograms\\
%xx   \verb+\SI{34}{\m}+&  \SI{34}{\m}& 34 meters\\
%xx   % The next input line is too wide for the margins
%xx   % so I'm splitting it into pieces.
%xx   \verb+\SI{4.5e3}{\kg\per\m\squared}+&
%xx     \SI{4.5e3}{\kg\per\m\squared}&
%xx     \(= \num{4.5e3}\,\si{\kg}/\si{\m\squared}\)\\
%xx \end{tabular}
%xx \end{VerbatimOut}
%xx
%xx \MyIO
%xx
%xx \begin{VerbatimOut}{z.out}
%xx
%xx How many seconds are in a non-leap year that does not have any leap seconds?
%xx % I tried several things and couold not get \cancel to work with \per.
%xx % Mark Senn    2019-12-29
%xx \begin{align*}
%xx            \frac{\SI{365}{\cancel\d}}{\si{\y}}
%xx     \times \frac{\SI{24}{\cancel\h}}{\si{\cancel\d}}
%xx     \times \frac{\SI{60}{\cancel\min}}{\si{\cancel\h}}
%xx     \times \frac{\SI{60}{\s}}{\si{\cancel\min}}
%xx     % From http://www.emerson.emory.edu/services/latex/latex_119.html
%xx     %     Spacing in Math Mode
%xx     %     In a math environment, LaTeX ignores the spaces you type
%xx     %     and puts in the spacing that it thinks is best. LaTeX formats
%xx     %     mathematics the way it's done in mathematics texts. If you
%xx     %     want different spacing, LaTeX provides the following four
%xx     %     commands for use in math mode:
%xx     %         \; - a thick space
%xx     %         \: - a medium space
%xx     %         \, - a thin space
%xx     %         \! - a negative thin space
%xx     & = \num{31536000}\;\frac{\si{\s}}{\si{\y}}\\
%xx     & = \SI{31536000}{\s\per\y}\\
%xx     & \approx \SI{3e7}{\s\per\y}\\
%xx     & \approx \text{30 million\,}\si{\s\per\y}\\
%xx \end{align*}
%xx \end{VerbatimOut}
%xx
%xx \MyIO
%xx
%xx % Information in this section is consistent with [SIUNITX-20220607].
%xx % Checked on 2022-06-17.



\begin{VerbatimOut}{z.out}


\section{Numbers}
\end{VerbatimOut}

\MyIO


\begin{VerbatimOut}{z.out}

\subsection{Number Examples}

\begin{inlinetable}
  \begin{tabular}{@{}lll@{}}
    \toprule
    \textbf{Input}& \textbf{Output}& \textbf{Comment}\\
    \midrule
    |\num{3.45d-4}|&         \num{3.45d-4}\\
    |\num{-0.12345}|&        \num{-0.12345}&               note the small space after the |3|\\
    |\num{-0.1234}|&         \num{-0.1234}&                note no space between the |3| and |4|\\
    |\num{-.123}|&           \num{-.123}&                  the |0.| is inserted automatically\\
    |\num{.123}|&            \num{.123}&                   the |0.| is inserted automatically\\
    |\num{0.123}|&           \num{0.123}\\
    |\num{0.1234}|&          \num{0.1234}&                 note no space between the |3| and |4|\\
    |\num{0.12345}|&         \num{0.12345}&                note the small space after the |3|\\
    |\num{123}|&             \num{123}\\
    |\num{1234}|&            \num{1234}\\
    |\num{12345}|&           \num{12345}&                  note the small space after the |2|\\
    |\num{2e4}|&             \num{2e4}\\
    |\num{e5}|&              \num{e5}\\
    |\num{2.34567e6}|&       \num{2.34567e6}&              note the small space after the |5|\\[6pt]
    |\numlist{10;30;50;70}|& \numlist{10;30;50;70}\\[6pt]
    |\numproduct{10 x 20}|&  \numproduct{10 x 20}\\[6pt]
    |\numrange{10}{30}|&     \numrange{10}{30}\\[6pt]
    |\complexnum{-4}|&       \complexnum{-4}\\
    |\complexnum{-3-2i}|&    \complexnum{-3-2i}\\
    |\complexnum{2+3i}|&     \complexnum{2+3i}\\
    |\complexnum{4+-5i}|&    \complexnum{4+-5i}\\
    \bottomrule
  \end{tabular}
\end{inlinetable}
\end{VerbatimOut}

\MyIO


\begin{VerbatimOut}{z.out}


\section{Units}

A unit consists of an optional prefix
followed by a unit.
\end{VerbatimOut}

\MyIO


\begin{VerbatimOut}{z.out}

\subsection{Binary Prefixes}

These binary prefixes are defined by \siunitxLogo\ and
are used for units related to computers,
data storage,
data transmission,
etc.
  
\begin{inlinetable}
  % Define a math mode version of "r" column type.
  \newcolumntype{m}{>{$}l<{$}}%
  \begin{tabular}{@{}mllll@{}}
    \toprule
    \textbf{Factor}& \textbf{Prefix}& \textbf{Pronounciation}& \textbf{Command}& \textbf{Symbol}\\
    \midrule
    2^{10}& kibi& kilobee& |\kibi|& \unit{\kibi\nounit}\\
    2^{20}& mebi& megabee& |\mebi|& \unit{\mebi\nounit}\\
    2^{30}& gibi& gigabee& |\gibi|& \unit{\gibi\nounit}\\
    2^{40}& tebi& terabee& |\tebi|& \unit{\tebi\nounit}\\
    2^{50}& pebi& pebibee& |\pebi|& \unit{\pebi\nounit}\\
    2^{60}& exbi& exbibee& |\exbi|& \unit{\exbi\nounit}\\
    2^{70}& zebi& zetabee& |\zebi|& \unit{\zebi\nounit}\\
    2^{80}& yobi& yobibee& |\yobi|& \unit{\yobi\nounit}\\
    \bottomrule
  \end{tabular}
\end{inlinetable}
\end{VerbatimOut}

\MyIO


\begin{VerbatimOut}{z.out}

\subsection{Decimal Prefixes}

These decimal prefixes are used for units that are usually not
related to computers.

\begin{inlinetable}
  \newcolumntype{m}{>{$}l<{$}}%
  \begin{tabular}{@{}mlll@{}}
    \toprule
    \textbf{Factor}& \textbf{Prefix}& \textbf{Command}& \textbf{Symbol}\\
    \midrule
    10^{-24}& yocto& |\yocto|& \unit{\yocto\nounit}\\
    10^{-21}& zepto& |\zepto|& \unit{\zepto\nounit}\\
    10^{-18}& atto&  |\atto|&  \unit{\atto\nounit}\\
    10^{-15}& femto& |\femto|& \unit{\femto\nounit}\\
    10^{-12}& pico&  |\pico|&  \unit{\pico\nounit}\\
    10^{ -9}& nano&  |\nano|&  \unit{\nano\nounit}\\
    10^{ -6}& micro& |\micro|& \unit{\micro\nounit}\\
    10^{ -3}& milli& |\milli|& \unit{\milli\nounit}\\
    10^{ -2}& centi& |\centi|& \unit{\centi\nounit}\\
    10^{ -1}& deci&  |\deci|&  \unit{\deci\nounit}\\
    10^{  1}& deca&  |\deca|&  \unit{\deca\nounit}\\   % or \deka
    10^{  2}& hecto& |\hecto|& \unit{\hecto\nounit}\\
    10^{  3}& kilo&  |\kilo|&  \unit{\kilo\nounit}\\
    10^{  6}& mega&  |\mega|&  \unit{\mega\nounit}\\
    10^{  9}& giga&  |\giga|&  \unit{\giga\nounit}\\
    10^{ 12}& tera&  |\tera|&  \unit{\tera\nounit}\\
    10^{ 15}& peta&  |\peta|&  \unit{\peta\nounit}\\
    10^{ 18}& exa&   |\exa|&   \unit{\exa\nounit}\\
    10^{ 21}& zetta& |\zetta|& \unit{\zetta\nounit}\\
    10^{ 24}& yotta& |\yotta|& \unit{\yotta\nounit}\\
    \bottomrule
  \end{tabular}
\end{inlinetable}
\end{VerbatimOut}

\MyIO


\begin{VerbatimOut}{z.out}

Here is the complete list of units defined
by the \siunitxLogo\ package:

{
  \ZZbaselinestretch{1}
  % See
  %     https://tex.stackexchange.com/questions/100351/siunitxs-micro-symbol-has-serif
  % for solution to \micro using wrong font.
  \newcommand{\q}{\quad}
  \begin{longtable}{@{}lllll@{}}%
      \caption{Units}\\
      \textbf{Name}& \textbf{Input}& \textbf{Output}& \textbf{Input}& \textbf{Output}\\[6pt]
    \endfirsthead
      \caption[]{~\emph{continued}}\\
      \textbf{Name}& \textbf{Input}& \textbf{Output}& \textbf{Input}& \textbf{Output}\\[6pt]
    \endhead
      \multicolumn{4}{@{}c@{}}{\emph{continued on next page}}%
    \endfoot
    \endlastfoot
    % Normally I'd indent the following lines but that makes them
    % to wide to print the input within the margins.
    ampere&               |\A|&    \unit{\A}&    |\ampere|&             \unit{\ampere}\\
    \q picoampere&        |\pA|&   \unit{\pA}&   |\pico\ampere|&        \unit{\pico\ampere}\\
    \q nanoampere&        |\nA|&   \unit{\nA}&   |\nano\ampere|&        \unit{\nano\ampere}\\
    \q microampere&       |\uA|&   \unit{\uA}&   |\micro\ampere|&       \unit{\micro\ampere}\\
    \q milliampere&       |\mA|&   \unit{\mA}&   |\milli\ampere|&       \unit{\milli\ampere}\\
    \q kiloampere&        |\kA|&   \unit{\kA}&   |\kilo\ampere|&        \unit{\kilo\ampere}\\[6pt]
    astronomical unit&    &        &             |\astronomicalunit|&   \unit{\astronomicalunit}\\[6pt]
    becquerel&            &        &             |\becquerel|&          \unit{\becquerel}\\[6pt]
    bel&                  &        &             |\bel|&                \unit{\bel}\\
    \q decibel&           |\dB|&   \unit{\dB}&   |\decibel|&            \unit{\decibel}\\[6pt]
    bit&                  &        &             |\bit|&                \unit{\bit}\\[6pt]
    byte&                 &        &             |\byte|&               \unit{\byte}\\[6pt]
    candela&              &        &             |\candela|&            \unit{\candela}\\[6pt]
    coulomb&              |\C|&    \unit{\C}&    |\coulomb|&            \unit{\coulomb}\\
    \q nanocoulomb&       |\nC|&   \unit{\nC}&   |\nano\coulomb|&       \unit{\nano\coulomb}\\
    \q microcoulomb&      |\uC|&   \unit{\uC}&   |\micro\coulomb|&      \unit{\micro\coulomb}\\
    \q millicoulomb&      |\mC|&   \unit{\mC}&   |\milli\coulomb|&      \unit{\milli\coulomb}\\[6pt]
    dalton&               &        &             |\dalton|&             \unit{\dalton}\\[6pt]
    day&                  &        &             |\day|&                \unit{\day}\\[6pt]
    degree (plane angle)& &        &             |\degree|&             \unit{\degree}\\
    \q minute&            &        &             |\arcminute|&          \unit{\arcminute}\\
    \q second&            &        &             |\arcsecond|&          \unit{\arcsecond}\\[6pt]
    degree Celsius&       &        &             |\degreeCelsuis|&      \unit{\degreeCelsius}\\[6pt]
    electronvolt&         |\eV|&   \unit{\eV}&   |\electronvolt|&       \unit{\electronvolt}\\
    \q millielectronvolt& |\meV|&  \unit{\meV}&  |\milli\electronvolt|& \unit{\milli\electronvolt}\\
    \q kiloelectronvolt&  |\keV|&  \unit{\keV}&  |\kilo\electronvolt|&  \unit{\kilo\electronvolt}\\
    \q megaelectronvolt&  |\MeV|&  \unit{\MeV}&  |\mega\electronvolt|&  \unit{\mega\electronvolt}\\
    \q gigaelectronvolt&  |\GeV|&  \unit{\GeV}&  |\giga\electronvolt|&  \unit{\giga\electronvolt}\\
    \q teraelectronvolt&  |\TeV|&  \unit{\TeV}&  |\tera\electronvolt|&  \unit{\tera\electronvolt}\\[6pt]
    farad&                |\F|&    \unit{\F}&    |\farad|&              \unit{\farad}\\
    \q femtofarad&        |\fF|&   \unit{\fF}&   |\femto\farad|&        \unit{\femto\farad}\\
    \q picofarad&         |\pF|&   \unit{\pF}&   |\pico\farad|&         \unit{\pico\farad}\\
    \q nanofarad&         |\nF|&   \unit{\nF}&   |\nano\farad|&         \unit{\nano\farad}\\
    \q microfarad&        |\uF|&   \unit{\uF}&   |\micro\farad|&        \unit{\micro\farad}\\[6pt]
    gram&                 |\g|&    \unit{\g}&    |\gram|&               \unit{\gram}\\
    \q femtogram&         |\fg|&   \unit{\fg}&   |\femto\gram|&         \unit{\femto\gram}\\
    \q picogram&          |\pg|&   \unit{\pg}&   |\pico\gram|&          \unit{\pico\gram}\\
    \q nanogram&          |\ng|&   \unit{\ng}&   |\nano\gram|&          \unit{\nano\gram}\\
    \q microgram&         |\ug|&   \unit{\ug}&   |\micro\gram|&         \unit{\micro\gram}\\
    \q milligram&         |\mg|&   \unit{\mg}&   |\milli\gram|&         \unit{\milli\gram}\\
    \q kilogram&          |\kg|&   \unit{\kg}&   |\kilogram|&           \unit{\kilogram}\\
    &                     &        &             |\kilo\gram|&          \unit{\kilo\gram}\\[6pt]
    gray&                 &        &             |\gray|&               \unit{\gray}\\[6pt]
    hectare&              &        &             |\hectare|&            \unit{\hectare}\\[6pt]
    henry&                |\H|&    \unit{\H}&    |\henry|&              \unit{\henry}\\
    \q femtohentry&       |\fH|&   \unit{\fH}&   |\femto\henry|&        \unit{\femto\henry}\\
    \q picohentry&        |\pH|&   \unit{\pH}&   |\pico\henry|&         \unit{\pico\henry}\\
    \q nanohentry&        |\nH|&   \unit{\nH}&   |\nano\henry|&         \unit{\nano\henry}\\
    \q microhentry&       |\uH|&   \unit{\uH}&   |\micro\henry|&        \unit{\micro\henry}\\
    \q millihentry&       |\mH|&   \unit{\mH}&   |\milli\henry|&        \unit{\milli\henry}\\[6pt]
    hertz&                |\Hz|&   \unit{\Hz}&   |\hertz|&              \unit{\hertz}\\
    \q millihertz&        |\mHz|&  \unit{\mHz}&  |\milli\hertz|&        \unit{\milli\hertz}\\
    \q kilohertz&         |\kHz|&  \unit{\kHz}&  |\kilo\hertz|&         \unit{\kilo\hertz}\\
    \q megahertz&         |\MHz|&  \unit{\MHz}&  |\mega\hertz|&         \unit{\mega\hertz}\\
    \q gigahertz&         |\GHz|&  \unit{\GHz}&  |\giga\hertz|&         \unit{\giga\hertz}\\
    \q terahertz&         |\THz|&  \unit{\THz}&  |\tera\hertz|&         \unit{\tera\hertz}\\[6pt]
    hour&                 &        &             |\hour|&               \unit{\hour}\\[6pt]
    joule&                |\J|&    \unit{\J}&    |\joule|&              \unit{\joule}\\
    \q microjoule&        |\uJ|&   \unit{\uJ}&   |\micro\joule|&        \unit{\micro\joule}\\
    \q millijoule&        |\mJ|&   \unit{\mJ}&   |\milli\joule|&        \unit{\milli\joule}\\
    \q kilojoule&         |\kJ|&   \unit{\kJ}&   |\kilo\joule|&         \unit{\kilo\joule}\\
    \q megajoule&         &        &             |\mega\joule|&         \unit{\mega\joule}\\[6pt]
    katal&                &        &             |\katal|&              \unit{\katal}\\[6pt]
    kelvin&               |\K|&    \unit{\K}&    |\kelvin|&             \unit{\kelvin}\\[6pt]
    kilowatt hour&        |\kWh|&  \unit{\kWh}&  \\[6pt]
    liter&                |\L|&    \unit{\L}&    |\liter|&              \unit{\liter}\\
    \q microliter&        |\uL|&   \unit{\uL}&   |\micro\liter|&        \unit{\micro\liter}\\
    \q milliliter&        |\mL|&   \unit{\mL}&   |\milli\liter|&        \unit{\milli\liter}\\
    \q hectoliter&        |\hL|&   \unit{\hL}&   |\hecto\liter|&        \unit{\hecto\liter}\\[6pt]
    lumen&                &        &             |\lumen|&              \unit{\lumen}\\[6pt]
    lux&                  &        &             |\lux|&                \unit{\lux}\\[6pt]
    meter&                |\m|&    \unit{\m}&    |\meter|&              \unit{\meter}\\
    \q picometer&         |\pm|&   \unit{\pm}&   |\pico\meter|&         \unit{\pico\meter}\\
    \q nanometer&         |\nm|&   \unit{\nm}&   |\nano\meter|&         \unit{\nano\meter}\\
    \q micrometer&        |\um|&   \unit{\um}&   |\micro\meter|&        \unit{\micro\meter}\\
    \q millimeter&        |\mm|&   \unit{\mm}&   |\milli\meter|&        \unit{\milli\meter}\\
    \q centimeter&        |\cm|&   \unit{\cm}&   |\centi\meter|&        \unit{\centi\meter}\\
    \q decimeter&         |\dm|&   \unit{\dm}&   |\deci\meter|&         \unit{\deci\meter}\\
    \q kilometer&         |\km|&   \unit{\km}&   |\kilo\meter|&         \unit{\kilo\meter}\\[6pt]
    minute (time)&        |\min|&  \unit{\min}&  |\minute|&             \unit{\minute}\\[6pt]
    mole&                 |\mol|&  \unit{\mol}&  |\mole|&               \unit{\mole}\\
    \q femtomole&         |\fmol|& \unit{\fmol}& |\femto\mole|&         \unit{\femto\mole}\\
    \q picomole&          |\pmol|& \unit{\pmol}& |\pico\mole|&          \unit{\pico\mole}\\
    \q nanomole&          |\nmol|& \unit{\nmol}& |\nano\mole|&          \unit{\nano\mole}\\
    \q micromole&         |\umol|& \unit{\umol}& |\micro\mole|&         \unit{\micro\mole}\\
    \q millimole&         |\mmol|& \unit{\mmol}& |\milli\mole|&         \unit{\milli\mole}\\
    \q kilomole&          |\kmol|& \unit{\kmol}& |\kilo\mole|&          \unit{\kilo\mole}\\[6pt]
    neper&                &        &             |\neper|&              \unit{\neper}\\[6pt]
    newton&               |\N|&    \unit{\N}&    |\newton|&             \unit{\newton}\\
    \q millinewton&       |\mN|&   \unit{\mN}&   |\milli\newton|&       \unit{\milli\newton}\\
    \q kilonewton&        |\kN|&   \unit{\kN}&   |\kilo\newton|&        \unit{\kilo\newton}\\
    \q meganewton&        |\MN|&   \unit{\MN}&   |\mega\newton|&        \unit{\mega\newton}\\[6pt]
    ohm&                  &        &             |\ohm|&                \unit{\ohm}\\
    \q milliohm&          |\mohm|& \unit{\mohm}& |\milli\ohm|&          \unit{\milli\ohm}\\
    \q kiloohm&           |\kohm|& \unit{\kohm}& |\kilo\ohm|&           \unit{\kilo\ohm}\\
    \q megaohm&           |\Mohm|& \unit{\Mohm}& |\mega\ohm|&           \unit{\mega\ohm}\\[6pt]
    pascal&               |\Pa|&   \unit{\Pa}&   |\pascal|&             \unit{\pascal}\\
    \q kilopascal&        |\kPa|&  \unit{\kPa}&  |\kilo\pascal|&        \unit{\kilo\pascal}\\
    \q megapascal&        |\MPa|&  \unit{\MPa}&  |\mega\pascal|&        \unit{\mega\pascal}\\
    \q gigapascal&        |\GPa|&  \unit{\GPa}&  |\giga\pascal|&        \unit{\giga\pascal}\\[6pt]
    percent&              &        &             |\percent|&            \unit{\percent}\\[6pt]
    radian&               &        &             |\radian|&             \unit{\radian}\\[6pt]
    second&               |\s|&    \unit{\s}&    |\second|&             \unit{\second}\\
    \q attosecond&        |\as|&   \unit{\as}&   |\atto\second|&        \unit{\as}\\
    \q femtosecond&       |\fs|&   \unit{\fs}&   |\femto\second|&       \unit{\fs}\\
    \q picosecond&        |\ps|&   \unit{\ps}&   |\pico\second|&        \unit{\ps}\\
    \q nanosecond&        |\ns|&   \unit{\ns}&   |\nano\second|&        \unit{\ns}\\
    \q microsecond&       |\us|&   \unit{\us}&   |\micro\second|&       \unit{\us}\\
    \q millisecond&       |\ms|&   \unit{\ms}&   |\milli\second|&       \unit{\ms}\\[6pt]
    siemens&              &        &             |\siemens|&            \unit{\siemens}\\[6pt]
    sievert&              &        &             |\sievert|&            \unit{\sievert}\\[6pt]
    steradian&            &        &             |\steradian|&          \unit{\steradian}\\[6pt]
    tesla&                &        &             |\tesla|&              \unit{\tesla}\\[6pt]
    tonne (metric ton)&   &        &             |\tonne|&              \unit{\tonne}\\[6pt]
    volt&                 |\V|&    \unit{\V}&    |\volt|&               \unit{\volt}\\
    \q picovolt&          |\pV|&   \unit{\pV}&   |\pico\volt|&          \unit{\pico\volt}\\
    \q nanovolt&          |\nV|&   \unit{\nV}&   |\nano\volt|&          \unit{\nano\volt}\\
    \q microvolt&         |\uV|&   \unit{\uV}&   |\micro\volt|&         \unit{\micro\volt}\\
    \q millivolt&         |\mV|&   \unit{\mV}&   |\milli\volt|&         \unit{\milli\volt}\\
    \q kilovolt&          |\kV|&   \unit{\kV}&   |\kilo\volt|&          \unit{\kilo\volt}\\[6pt]
    watt&                 |\W|&    \unit{\W}&    |\watt|&               \unit{\watt}\\
    \q nanowatt&          |\nW|&   \unit{\nW}&   |\nW|&                 \unit{\nano\watt}\\
    \q microwatt&         |\uW|&   \unit{\uW}&   |\micro\watt|&        \unit{\micro\watt}\\
    \q milliwatt&         |\mW|&   \unit{\mW}&   |\milli\watt|&        \unit{\milli\watt}\\
    \q kilowatt&          |\kW|&   \unit{\kW}&   |\kilo\watt|&         \unit{\kilo\watt}\\
    \q megawatt&          |\MW|&   \unit{\MW}&   |\mega\watt|&         \unit{\mega\watt}\\
    \q gigawatt&          |\GW|&   \unit{\GW}&   |\giga\watt|&         \unit{\giga\watt}\\[6pt]
    weber&                &        &             |\weber|&             \unit{\weber}\\
  \end{longtable}
}
\end{VerbatimOut}

\MyIO


\begin{VerbatimOut}{z.out}

\subsection{Unit Examples}

\begin{inlinetable}
  \begin{tabular}{@{}ll@{}}
    \toprule
    \textbf{Input}& \textbf{Output}\\
    \midrule
    |\unit{\meter}|&       \unit{\meter}\\
    |\unit{\m}|&           \unit{\m}\\
    |\unit{m}|&            \unit{m}\\[6pt]
    |\unit{\milli\meter}|& \unit{\milli\meter}\\
    |\unit{\mm}|&          \unit{\mm}\\
    |\unit{mm}|&           \unit{mm}\\
    \bottomrule
  \end{tabular}
\end{inlinetable}
\end{VerbatimOut}

\MyIO


\begin{VerbatimOut}{z.out}
  
\subsection{Defining your own unit}

% Define \c as a unit for the speed of light.
\DeclareSIUnit{\c}{\text{c}}
The speed of light in a vacuum (\unit{\c}) is defined as \qty{299792457}{m\per s}
\cite{wikipedia-speed-of-light}.
\end{VerbatimOut}

\MyIO


\begin{VerbatimOut}{z.out}


\section{Combined Unit}

Units can be combined to make a combined unit.
|\NE| (negative exponent)
and |\PE| (positive exponent)
are defined in PurdueThesis.cls.
\end{VerbatimOut}

\MyIO


\begin{VerbatimOut}{z.out}

\subsection{Combined Unit Examples}

\begin{inlinetable}
  \begin{tabular}{@{}ll@{}}
    \toprule
    \textbf{Input}& \textbf{Output}\\
    \midrule
    |\unit{\meter\per\second}|&   \unit{\meter\per\second}\\
    |\unit{\m\per\s}|&            \unit{\m\per\s}\\
    |\unit{m.s^{-1}}|&            \unit{m.s^{-1}}\\
    |\unit{m.s\NE1}|&             \unit{m.s\NE1}\\[6pt]
    |\unit{m.s^2}|&               \unit{m.s^2}\\
    |\unit{m.s\PE2}|&             \unit{m.s\PE2}\\
    \bottomrule
  \end{tabular}
\end{inlinetable}
\end{VerbatimOut}

\MyIO


\begin{VerbatimOut}{z.out}


\section{Quantity}

A quantity consists of a number with a unit or combined units.

\subsection{Quantity Examples}

\begin{inlinetable}
  \begin{tabular}{@{}ll@{}}
    \toprule
    \textbf{Input}& \textbf{Output}\\
    \midrule
    |\qty{2}{\meter}|&            \qty{2}{\meter}\\
    |\qty{3}{\m}|&                \qty{3}{\m}\\
    |\qty{4}{m}|&                 \qty{4}{m}\\[6pt]
    |\qty{2}{\meter\per\second}|& \qty{2}{\meter\per\second}\\
    |\qty{3}{\m\per\s}|&          \qty{3}{\m\per\s}\\
    |\qty{4}{m.s^{-1}}|&          \qty{4}{m.s^{-1}}\\
    |\qty{5}{m.s\NE1}|&           \qty{5}{m.s\NE1}\\
    \bottomrule
  \end{tabular}
\end{inlinetable}
\end{VerbatimOut}

\MyIO


\begin{VerbatimOut}{z.out}

Using text math mode
|\(|\ldots|\)|
so spacing around equal signs is correct:\\
\(\qty{1}{W} = \qty{1}{J.s\NE1} = \qty{1}{kg.m\PE2.s\NE3}\)
\end{VerbatimOut}

\MyIO


\begin{VerbatimOut}{z.out}

The gravitation constant,
\(G\),
is approximately
\qty{6.674e-11}{m\PE3.kg\NE1.s\NE2}
\cite{wikipedia-gravitational-constant}.
\end{VerbatimOut}

\MyIO


\begin{VerbatimOut}{z.out}

The earth's gravitation acceleration is
\qty{9.80655}{m.s\NE2}
\cite{wikipedia-gravitational-acceleration}.
\end{VerbatimOut}

\MyIO


\begin{VerbatimOut}{z.out}


\section{Etc.}

\subsection{Angles}

\begin{inlinetable}
  \begin{tabular}{@{}lll@{}}
    \toprule
    \textbf{Input}& \textbf{Output}& \textbf{Comment}\\
    \midrule
    |-\ang{10}|&    -\ang{10}\\
    |\ang{;;3}|&    \ang{;;3}\\
    |\ang{;2;}|&    \ang{;2}\\
    |\ang{1}|&      \ang{1}\\
    |\ang{1;2}|&    \ang{1;2}\\
    |\ang{1;2;3}|&  \ang{1;2;3}\\
    |\ang{10}|&     \ang{10}\\
    \bottomrule
  \end{tabular}
\end{inlinetable}
\end{VerbatimOut}

\MyIO


\begin{VerbatimOut}{z.out}

\subsection{Cancel}

Using |\textstyle| size:
\unit[per-mode = fraction]
{\cancel\kilogram\meter\per\cancel\kilogram\per\second}
\end{VerbatimOut}

\MyIO


\begin{VerbatimOut}{z.out}

Using |\displaystyle| size:
{
  \(
    \displaystyle
    \unit[per-mode = fraction]%
    {\cancel\kilogram\meter\per\cancel\kilogram\per\second}
  \)
}
\end{VerbatimOut}

\MyIO


\begin{VerbatimOut}{z.out}
  
I first tried to do this using the \siunitxLogo\ |\per| command
but couldn't figure it out.
I decided to demonstrate how to do this without using SI unit abbreviations.
\begin{align}
          \frac {365\,\cancel{\textrm{day}}}    {\textrm{year}}
  \times  \frac {24\,\cancel{\textrm{hour}}}    {\cancel{\textrm{day}}}
  \times  \frac {60\,\cancel{\textrm{minute}}}  {\cancel{\textrm{hour}}}
  \times  \frac {60\,\textrm{\textrm{second}}}  {\cancel{\textrm{minute}}}
  & = 31,536,000 \textrm{ seconds/year}\nonumber\\
  & = \qty{31536000}{s.y\NE1}\quad\text{(in SI units)}\nonumber\\
  & \approx \textrm{32 million seconds/year}\nonumber
\end{align}
\end{VerbatimOut}

\MyIO


% TODO:
% Figure out how to format this and add it to the document.
% To estimate the number of digits in the answer:
% 365 -> 3 count all digits in first number
% 24 -> 1  count all digits and subtract one if first digit is 1 through 4
% 60 -> 2
% 60 -> 2


% The International System of Units (SI)
% https://www.bipm.org/en/measurement-units/

% The International System of Units (SI): Defining constants
% https://www.bipm.org/en/measurement-units/si-defining-constants

% The International System of Units (SI): Base units
% https://www.bipm.org/en/measurement-units/si-base-units


%xx \begin{VerbatimOut}{z.out}
%xx
%xx {%
%xx   \ZZbaselinestretch{1}
%xx   \newcommand\vsp{\noalign{\vspace*{6pt}}}
%xx   % From
%xx   % https://tex.stackexchange.com/questions/31508/flushleft-with-p-option-in-tabular
%xx   %     It's necessary to use the \arraybackslash in the last column,
%xx   %     otherwise \\ would not end the table row.  You can use \newline
%xx   %     to end lines in the last column cells (and the regular \\ in
%xx   %     the other column cells).
%xx   %     ...
%xx   %     If you need it often, consider defining a new column type using
%xx   %     array features, as I did here:
%xx   %         \newcolumntype{P}[1]{>{\raggedright\arraybackslash}p{#1}}
%xx   \newcolumntype{P}[1]{>{\raggedright\arraybackslash}p{#1}}%
%xx % \begin{longtable}{@{}P{1.4in}P{1in}llP{1.8in}@{}}
%xx % \begin{longtable}{@{}P{1in}P{1in}llP{1.8in}@{}}
%xx % \begin{longtable}{@{}P{1.2in}P{1in}llP{1.8in}@{}}
%xx % \begin{longtable}{@{}P{90.72pt}P{1in}llP{1.8in}@{}}  % 1.2in (86.72pt) + 4pt = 90.72pt
%xx   \begin{longtable}{@{}P{1.4in}P{1in}llP{1.8in}@{}}% 1.2in (86.72pt) + 4pt = 90.72pt
%xx     % \aa ngstr\"om&
%xx     %   length&
%xx     %   \si{\AA}&
%xx     %   \verb+\si{\AA}+&
%xx     %   \SI{e-10}{\m}\\
%xx     bar&
%xx       pressure&
%xx       \si{\bar}&
%xx       \verb+\si{\bar}+&
%xx       \SI{e-5}{\Pa}\\
%xx     \quad millibar&
%xx       \ditto&
%xx       \si{\mbar}&
%xx       \verb+\si{\mbar}+&
%xx       \SI{e-3}{\bar}\\
%xx     speed of light&
%xx       speed&
%xx       \si{\clight}&
%xx       \verb+\si{\clight}+&
%xx       \SI{299792458}{\m\per\s}\\
%xx %   temperatures
%xx %
%xx %       \degreeCelsius
%xx %       \celsius
%xx %
%xx %       range
%xx %           \SIrange{1}{5}{\metre}
%xx %           \SIrange{1}{5}{\milli\metre}
%xx %
%xx %
%xx %   numbers
%xx %   -10^{10}           \num{-e10}
%xx %                      \num{12345.67890}
%xx %                      \num{1+-2i}
%xx %                      \num{.3.45}
%xx %
%xx %       \celsius
