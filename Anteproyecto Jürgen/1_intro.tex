\begin{doublespace}
\begin{tightcenter}
\section{1. Introducción}
\mylinespacing
\end{tightcenter}

La avanzada época de la computación nos ha llevado a un presente donde las necesidades de procesamiento crecen constantemente, especialmente en el campo de la investigación, donde cada día se requiere más que solo teoría. Aún así en la Universidad del Valle no existen, de forma accesible, medios con los que ejecutar trabajos con requerimientos computacionales extensos.

\vspace{3mm} 

En nuestra alma mater existen recursos computacionales subutilizados, los cuales podrían ser aprovechados para crear un servicio computacional con gran poder de procesamiento dispuesto para cálculos con fines académicos. Esto resultaría de gran utilidad para la comunidad universitaria, ya que permitiría la realización de nuevos trabajos investigativos que actualmente no son posibles por la carencia de un medio con estas capacidades.

\vspace{3mm} 

Estos recursos son los computadores de la Sala \textit{Jürgen Tischer}, elementos normalmente inutilizados en los periodos en los que no hay clase y el clúster computacional \textit{Bochica}, actualmente en desuso. Ambos bienes pertenecen al Departamento de Matemáticas de la Universidad del Valle.

\vspace{3mm} 

Este proyecto propone integrarlos e implementar un gestor de cola de tareas con capacidades de procesamiento paralelo y distribuido. De esta manera sería posible alcanzar una mayor capacidad de procesamiento y a su vez, facilitar su uso apropiado para la investigación y la docencia.


\mylinespacing
\mylinespacing
\begin{tightcenter}
\end{tightcenter}
\end{doublespace}
