\begin{doublespace}
\begin{abstract}
El clúster Bochica en conjunto con la sala de computación Jürgen Tischer, ambos bienes pertenecientes al departamento de Matemáticas de la Universidad del Valle, resultan en una cantidad considerable de medios computacionales, sin embargo, subutilizados. Por lo que en este proyecto se plantea una propuesta en la que puedan ser empleados con fines investigativos con la ayuda de un gestor de cola de tareas con capacidades de paralelización para que, a pesar de contener recursos heterogéneos, resulten beneficiosos por su capacidad total de cómputo.
\end{abstract}
\end{doublespace}
