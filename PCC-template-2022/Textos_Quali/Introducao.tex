\noindent Este capítulo busca apresentar o contexto no qual o presente projeto de doutorado está inserido, assim como os elementos que motivam o seu desenvolvimento, os argumentos que determinam a problemática e o seu objetivo.

\section{Contextualização}

Sistemas conversacionais são sistemas de software que possuem como característica principal a capacidade de interagir com seres humanos em linguagem natural. Para tanto, fazem uso de estratégias e técnicas de processamento de linguagem natural ou linguística computacional para emular a comunicação com humanos \cite{Tegos:2019}. Nesse sentido, esses mecanismos podem ser acionados por voz ou texto \cite{Dale:2016}, e visam funcionalidades específicas. Eles também são referidos por outras designações, tais como: \textit{chatbots}, \textit{chatterbot}, sistemas de diálogo, assistentes virtuais, assistentes digitais, agentes conversacionais\footnote{Neste projeto, os termos ``agente conversacional'' e ``sistema conversacional'' são adotados e utilizados em alternância, porque o autor assume que possuem significados idênticos.} e agentes virtuais \cite{Paschoal:2020FIE}.  Independente do vocábulo utilizado, são aplicações que se envolvem em diálogo com humano e podem ser úteis em muitos domínios que carecem de interações \cite{Montenegro:2019}. 

A interação fornecida pelos sistemas conversacionais não se limita a conversação. Embora ela seja a principal característica dessas aplicações, os sistemas podem emular determinadas características humanas, utilizando outros tipos de interação, tais como gestos, olhares e outras modalidades não verbais \cite{Montenegro:2019}.  Há sistemas conversacionais sendo estabelecidos com corpo, os quais recebem a denominação de agentes conversacionais incorporados, que se relacionam com usuários acenando a cabeça, alterando suas expressões faciais, realizando gestos \cite{cassell2000}. São produzidos com a intenção de explorar as linguagens corporais, de modo a trazer novos significados para a comunicação entre usuário-computador \cite{cassell2000}.

Explorar os sistemas conversacionais juntamente com suas modalidades de interação (verbais e não verbais) têm contribuído com o surgimento de sistemas com diferentes funcionalidades e sistemas para diferentes domínios. Há sistemas sendo estabelecidos para apoiar a área de negócios (\textit{e.g.}, \textit{e-commerce} e reserva de voos) \cite{Bavaresco:2020}, a área da saúde (\textit{e.g.}, cuidados de saúde e atendimento ao paciente) \cite{Car:2020,Montenegro:2019,Laranjo:2018}, o entretenimento (\textit{e.g.}, contar histórias e jogos) \cite{Sciuto:2018, Macias:2012} e, principalmente, o domínio da educação \cite{Paschoal:2020FIE, Winkler:2020}.

\citeonline{Tegos:2019} apontam que há um interesse crescente no desenvolvimento de sistemas conversacionais para o domínio da educação, dado a diversidade de situações existentes nesse contexto que necessitam de uma maior interação e suporte automatizado. %Os autores, ainda, argumentam que eles podem desempenhar diversas funções pedagógicas (\textit{e.g.}, tutores, companheiros de aprendizagem e \textit{coaches}). 
Em ambientes educacionais, podem ampliar (ou promover) a interação, porque têm a capacidade de conversar com os estudantes sobre um determinado assunto e assumir o papel de professor, aluno ou colega \cite{Moreno:2016}. Adicionalmente, podem ser utilizados para: coletar feedback dos alunos sobre %aspectos de 
um curso %que precisam ser melhorados 
\cite{Wambsganss:2020}, %estimular a presença social entre alunos \cite{krassmann2018},
reduzir o sentimento de isolamento social dos alunos \cite{krassmann2018}, %oferecer tutoria em tempo integral \cite{krassmann2018},
solucionar as dúvidas dos alunos imediatamente \cite{Moraes:2019}, fortalecer a compreensão dos alunos %do mesmo modo que os educadores
\cite{Winkler:2020}, aliviar problemas de engajamento  \cite{Winkler:2020}, aumentar a atenção e envolvimento dos alunos \cite{Moraes:2019}, ajudar os alunos a realizar atividades educacionais \cite{Winkler:2020}, levar a uma interação mais significativa \cite{Winkler:2020} e facilitar o processo de aprendizagem na ausência do professor \cite{Tegos:2020}.

Apesar do grande potencial que os sistemas conversacionais podem alcançar, o estado da arte tem oferecido pouco conhecimento experimental sobre o impacto desse tipo de sistema quando usado periodicamente em situações de aprendizagem \cite{paschoalframework}. Esse pouco conhecimento se deve a produção acadêmica da área que tem dado ênfase ao desenvolvimento desse sistema (\textit{e.g.}, codificação e modelagem de conhecimento do agente) \cite{io2017, bernardini2018, kuyven2018, roos2018}. Estudos que reportam experiências e lições aprendidas sobre a implantações e usos desses sistemas são escassos na literatura \cite{paschoalframework}. Isso, por sua vez, pode ser utilizado como premissa para questionar a qualidade dos sistemas que estão sendo produzidos e discutir se os sistemas conversacionais educacionais (ou pedagógicos) produzidos estão conseguindo atender com satisfação os seus objetivos.

A qualidade dos sistemas conversacionais tem sido considerada uma temática de pesquisa que precisa de maior atenção \cite{Wambsganss:2020, Ruane:2018, Ultes:2013, goh2007}. Com o surgimento da área, os estudos desenvolvidos se concentraram em produzir modelos de conhecimento, técnicas para recuperar informações e métodos para processar a língua natural humana \cite{abdul2015survey}. Ao passo que novas contribuições surgiam, a necessidade de avaliar e compreender melhor tais contribuições também emergiu \cite{Jain:2018, radziwill2017, abushawar2016}. Assim, abordagens para conferir se os sistemas conversacionais são capazes de interagir adequadamente ou apresentar o raciocínio esperado foram estabelecidas e refinadas \cite{Sugiyama2019, Voorhees2008}. Nesse contexto, o Teste de Turing\footnote{O Teste de Turing consiste em uma estratégia na qual uma máquina é submetida a uma avaliação, em que ela deve se passar por um humano e enganar um juiz humano, fazendo-o a acreditar que esteja conversando com um humano ao invés de um computador (ou sistema de software) \cite{Turing:1950}.} foi considerado como um exemplo precursor para avaliar o sucesso desse tipo de software, todavia, atualmente esse teste tem sido retratado como não adequado para os sistemas conversacionais modernos \cite{kaleem2016}. 

Recentemente, estudos têm proposto contribuições para apoiar a qualidade dos sistemas conversacionais, que variam desde abordagens para projetar, testar ou avaliar esse tipo de sistema de software \cite{deriu2020, liu:2016}. No entanto, o desafio em aberto é compreender o sucesso de um sistema conversacional específico \cite{kaleem2016}. O entendimento sobre o sucesso de um sistema conversacional vai além de garantir que ele consiga oferecer um diálogo com qualidade, porque envolve a construção de um entendimento sobre como os usuários interagem e percebem o sistema, assim como, se o sistema satisfaz os seus objetivos \cite{SCHMITT201512}. De acordo com \citeonline{Ruane:2018}, os usuários têm grandes expectativas para utilizar os sistemas conversacionais, portanto garantir qualidade é importante. Em particular, um aspecto que carece de contribuições tem relação com os fatores que influenciam na aceitação e sucesso do sistema conversacional em determinados cenários \cite{kaleem2016}.

Uma maneira de obter um entendimento concreto e baseado em evidências empíricas sobre esses fatores e conquistar uma melhor compreensão acerca da viabilidade e até mesmo qualidade dos sistemas conversacionais é por meio de estudos experimentais. No entanto, sistemas conversacionais definidos para o domínio educacional geralmente não têm sido avaliados por meio de estudos experimentais. Há diversos estudos que analisam a qualidade da conversação \cite{Paschoal:2019}, a usabilidade do sistema conversacional \cite{kim2020} e observaram as percepções dos alunos sobre o sistema \cite{krassmann2018}. Ao abordarem essas possibilidades de avaliações, os estudos se concentram em apenas um ou outro aspecto do sistema e deixam de contemplar atributos que são de fundamental importância \cite{paschoalframework}. Notavelmente, existem alguns estudos que exploram o conceito de experimentação, entretanto trabalhos recentes apontam que falta sistematização em tais estudos \cite{hobert2019, winkler2018}. 

A falta de sistematização em estudos experimentais pode prejudicar o entendimento sobre os procedimentos utilizados, atrapalhar a interpretação dos resultados e inviabilizar a replicação do estudo \cite{Wohlin, 493439}. Além disso, a ausência de sistematização pode ter relação com a falta de procedimentos que apoiam os pesquisadores a sistematizar o estudo experimental ou com a carência de conhecimento desses pesquisadores sobre a necessidade de se avaliar o sistema conversacional de modo sistematizado. Como os pesquisadores que trabalham com sistemas conversacionais na educação proveem de diferentes áreas \cite{hobert2019, winkler2018}, é possível que não haja um entendimento concreto sobre esse tipo de estudo. Isso explicaria a diversidade de métodos usados para compreender a qualidade de uma interação \cite{abushawar2016}, os diversos instrumentos definidos para coletar a percepção dos usuários \cite{norouzi2018}, dentre outros.

\section{Motivação}

Estudos experimentais têm sido utilizados na Engenharia de Software para validar técnicas e ferramentas que apoiam o desenvolvimento de sistemas de software. Eles podem ser explorados tanto na validação de tecnologias maduras quanto para evoluir tecnologias que não possuem tanta maturidade \cite{Shull}. Portanto, podem ser utilizados para reconhecer a eficácia de determinadas ferramentas e identificar problemas que existem nas ferramentas \cite{Vos:2012}. Nesse sentido, quando usados em contextos de ferramentas e ambientes de apoio ao desenvolvimento de software, também contribuem para garantia de qualidade de tais ambientes. No entanto, esse paradigma não pode ser abordado exclusivamente como um meio para validar uma tecnologia, mas como uma abordagem que ajuda as áreas do conhecimento a compreender e experimentar teorias, assim como pôr fim a mitos e crenças \cite{Devanbu}. 

De acordo com \citeonline{SOLARI201864}, em estudos experimentais a realidade é manipulada e observada de modo controlado, o que permite identificar fatores que afetam um determinado fenômeno. Ao serem produzidos com rigor, podem identificar benefícios e problemas que existem em mecanismos de apoio ao ensino, como os sistemas conversacionais educacionais, considerando os objetivos educacionais que se espera atingir a partir da disponibilidade e uso do sistema conversacional. Todavia, elaborar um estudo experimental com o rigor necessário não é trivial \cite{Vegas}. É preciso escolher as variáveis corretas, os tipos de dados certos e controlar os fatores existentes, dado que erros metodológicos invalidam as conclusões dos pesquisadores \cite{Kitchenham:2002}. Apesar de ser difícil e custoso, é uma abordagem necessária para amadurecer o conhecimento em uma temática \cite{Basili:1998}.

O estudo experimental precisa ser projetado, conduzido e reportado com rigor que garanta que os benefícios ou os problemas encontrados durante a avaliação, tenham sido resultantes do objeto que está sendo considerado no estudo \cite{493439, Wohlin}. Ao mesmo tempo, seus resultados devem ser reproduzíveis, em contextos equivalentes ou variados \cite{Carver:2014}. A reprodução de um experimento tem um importante significado para a experimentação, porque ajuda a determinar se os resultados de um experimento podem ser reproduzidos \cite{juristo2010}. Para que os resultados do estudo experimental sejam reproduzidos, o material gerado no decorrer do estudo experimental precisa estar disponível para a comunidade e os relatórios que contém a descrição do estudo devem exteriorizar todas as decisões tomadas, parâmetros estudados e objetos controlados \cite{SHEPPERD2018120}. 

Quando o estudo experimental oferece a comunidade todas as informações necessárias, os resultados observados como decorrência da sua execução agregarão um valor para a área, permitindo que estudos baseados em evidências, como revisões sistemáticas de literatura, possam comparar o estudo com outros da literatura \cite{Kitchenham:2004}. Esse tipo de análise é comum em áreas que tem histórico no uso do método empírico, como a Medicina. Na Medicina, o pesquisador busca compreender o corpo humano a fim de prever o seu comportamento frente a vários procedimentos e medicamentos \cite{493439}. Assim, as revisões sistemáticas podem auxiliar os médicos a obter as melhores evidências sobre a influência de um determinado medicamento no tratamento de pacientes, bem como, no julgamento clínico dos médicos especialistas \cite{Kitchenham:2004}. 

No contexto de sistemas conversacionais educacionais, pode-se explorar os estudos experimentais para ajudar a reconhecer qual o efeito promovido pelo agente conversacional em algum aspecto inerente ao aprendizado, como o ganho de aprendizado e a motivação do aluno. Ao mesmo tempo, o estudo pode envolver outros aspectos - abordados como variáveis no contexto da experimentação - como usabilidade do sistema, eficiência em oferecer respostas corretas ao aluno, dentre outros. Com o uso preciso de estudos experimentais em trabalhos desenvolvidos na área, futuramente, as revisões sistemáticas da literatura poderão obter evidências concretas sobre agentes conversacionais que conseguem promover a alfabetização, por exemplo.

A experimentação em sistemas conversacionais educacionais, ainda, ajudará a comunidade que está envolvida com a temática a compreender melhor os efeitos produzidos por tais sistemas. Ao se estudar experimentalmente o uso de um agente em um contexto de aprendizado, algumas certezas poderão ser contestadas e efeitos inesperados conseguirão ser descobertos. Por exemplo, quando o aluno é exposto para interagir com um agente que visa ajudá-lo a resolver uma atividade, o agente pode auxiliar o aluno a resolver a atividade ou apenas criar um sentimento no aluno que o faz acreditar que o ajudou, sem de fato ter contribuído para a aquisição de conhecimento. Sem uma análise cuidadosa, uma interpretação errônea pode ser feita pelos pesquisadores. Com o cuidado que exige a experimentação, a interpretação errônea pode ser evitada e mitigada. 

Explorar a experimentação no contexto de sistemas conversacionais, além de oferecer suporte para o pesquisador compreender melhor alguns aspectos de qualidade dos agentes conversacionais (em termos de validação de software), tem potencial para auxiliar a comunidade entender a repercussão do agente no processo de aprendizado. Perguntas como ``os agentes conversacionais que estão sendo estabelecidos para ajudar no ensino de engenharia de software são capazes de estimular o aluno a reconhecer a importância dessa disciplina?'' e ``os agentes conversacionais aumentam a autoconfiança dos estudantes de programação?'' poderão ter a oportunidade de serem investigadas e os resultados obtidos a partir dessas investigações, poderão ser úteis para futuras tomadas de decisões.

A utilidade do resultado obtido em um estudo experimental, por sua vez, depende da veracidade e segurança que o pesquisador responsável possui no processo utilizado no decorrer do estudo. De acordo com \citeonline{Shull}, se o pesquisador não tiver certeza que foi um determinado processo que produziu o resultado no estudo, o resultado não terá utilidade. Por isso, ao conduzir e reportar o estudo experimental o pesquisador precisa de cautela e ter um entendimento concreto sobre o seu objeto de estudo e processo usado, de modo a garantir que seu estudo não perca sua validade. Mesmo em áreas que tem um rico histórico em experimentação, como física e medicina, há relatos de estudos impossíveis de serem avaliados \cite{Kitchenham:2002}.

A validade do estudo pode ter relação com diferentes aspectos e muitas vezes podem ser associadas aos detalhes insuficientes apresentados. De acordo com \citeonline{vegas2016can} a confiabilidade dos resultados é altamente dependente do projeto e da qualidade do protocolo. Além disso, é preciso assegurar que os parâmetros estudados são relevantes, que os sujeitos participantes do experimento possuem o perfil adequado, que os resultados sejam generalizáveis \cite{Wright}. Outro aspecto que precisa de atenção remete aos testes estatísticos. A seleção inadequada de um teste prejudica o estudo, podendo até mesmo invalidar as descobertas descritas nos relatórios ou artigos científicos \cite{Kitchenham:2002}. Portanto, durante o estudo escolher os testes estatísticos corretos e garantir que a triagem precisa foi feita é de extrema relevância. 

Em algumas áreas da computação já existem mecanismos e artefatos que buscam guiar a condução de estudos experimentais e ajudam os pesquisadores a conduzir estudo com ameaças à validade mitigadas \cite{Wohlin}. A Engenharia de Software possui processos e frameworks metodológicos\footnote{Uma estrutura que agrega um conjunto de recursos que ajudam um pesquisador a definir e instanciar experimentos ou uma família de experimentos.} que oferecem diretrizes gerais e até mesmo orientações específicas para ajudar pesquisadores a melhorar processos, métodos e ferramentas de desenvolvimento \cite{1514443}. Apesar disso, há disciplinas que precisam de contribuições instanciadas e que a experimentação seja adaptada \cite{vegas2016can}. \citeonline{Vegas} complementa que não adianta copiar modelos de outras áreas, porque a experimentação precisa ser adaptada em disciplinas aplicadas. Como exemplos, ao longo dos últimos anos, foram definidos frameworks metodológicos para diferentes tópicos, como programação por pares \cite{Gallis:2003}, ensino de programação \cite{LilianTese:2019}, teste de software \cite{Vos:2012}, dentre outros.

No contexto de sistemas conversacionais educacionais, há relato recentes que os pesquisadores da área estão precisando de suporte metodológico instanciado \cite{hobert2019}. Até o momento, não há na literatura um esforço para especificar como os agentes conversacionais deste domínio precisam ser avaliados (de modo controlado), quais parâmetros precisam ser observados, fatores que precisam ser controlados, quais variáveis devem ser medidas e como elas devem ser mensuradas, quais ameaças que surgem e podem comprometer a validade do estudo, etc. Assim, acredita-se que um framework metodológico pode amparar a literatura, sendo também capaz ser utilizado para melhorar a qualidade de pesquisa sobre sistemas conversacionais na educação.

Buscar pelo apoio metodológico não somente é motivado pela discussão baseada na literatura, mas também por problemas com pesquisas que estão sendo realizadas no contexto do Laboratório de Engenharia de Software do Instituto de Ciências Matemáticas e de Computação da Universidade de São Paulo. Em particular, algumas dificuldades surgiram ao conduzir estudos experimentais no contexto do agente conversacional TOB--STT \cite{Paschoal:2019}, um mecanismo de apoio ao ensino de teste de software. O agente foi definido para ajudar alunos que estão estudando técnicas e critérios de teste. Estudos experimentais têm sido conduzidos para compreender seu impacto de situações de aprendizagem do domínio de teste, todavia a seleção de variáveis e fatores para serem controlados não tem sido trivial e muitas ameaças dificultam a generalização dos resultados. 

\section{Problema e hipótese de pesquisa}


Estudos que buscam avaliar os agentes conversacionais em contextos de aprendizagem não são recentes. Desde os primeiros trabalhos que conceberam agentes com objetos educacionais, os pesquisadores têm conduzido avaliações. Essas avaliações variam de estudo para estudo. Há estudos primários que abordam a aceitação do aluno com a tecnologia \cite{liu2019cbet, 9081419}, a satisfação do aluno \cite{krassmann2018}, estímulos motivacionais fortalecidos pelo agente \cite{FRYER2019279}, dentre outros. Em meio a diferentes aspectos que são avaliados, nota-se que não existe um procedimento sistemático para se conduzir avaliações, uma vez que cada pesquisador avalia o seu objeto de estudo da forma que acredita ser mais adequada, conforme seus costumes, usando suas próprias métricas e instrumentos para demonstrar a viabilidade de sua aplicação. De acordo com \citeonline{hobert2019} essa ausência de sistematização pode ter relação com a formação básica dos pesquisadores, que variam de diferentes áreas do conhecimento. O autor complementa descrevendo que os pesquisadores acabam avaliando os agentes conversacionais considerando hábitos que existem em suas áreas de formação e nem sempre têm conhecimento sobre a necessidade de uma sistematização.

Como os pesquisadores aplicam abordagens de avaliação de suas áreas de formação, ainda que o agente conversacional tenha como foco promover o aprendizado, nem sempre a eficácia de aprendizado promovido pelo agente é estudada/observada \cite{hobert2019}. Ainda, os pesquisadores tendem a avaliar aspectos particulares e, frequentemente, deixam de avaliar o agente por perspectivas diferentes que são de extrema importância. Isso demonstra que uma atenção maior precisa que ser oferecida. \citeonline{hobert2019} ao oferecer informações com base em uma revisão de literatura, indica que um suporte metodológico para apoiar o planejamento de avaliações de viabilidade precisa ser oferecido aos pesquisadores que estão propondo soluções baseadas em agentes conversacionais para o domínio da educação. 

\citeonline{winkler2018}, em uma revisão de literatura, observaram diferentes formas de avaliar a utilidade do agente conversacional educacional. Ao mesmo tempo, constataram que os estudos sobre agentes conversacionais na educação, em sua maioria, deixam de entender a influência do agente nos processos de ensino-aprendizagem dos estudantes. Normalmente, os estudos não abordam projetos que tentam mensurar, por exemplo, se o agente conversacional produz resultados significativos em termos de aprendizado. Desse modo, estudos adicionais devem investigar o valor do agente conversacional durante a aprendizagem que é mediada pelos agentes e quais as competências que podem ser suportadas pelos agentes. No entanto, para que essas investigações sejam feitas, de acordo com \citeonline{winkler2018}, é necessário apoio metodológico para estudos sistematizados, a fim de garantir que os resultados de aprendizagem sejam analisados. 

O suporte metodológico também precisa apoiar estudos que comparam o agente conversacional com outros objetos de estudos, dado que faltam estudos que confrontam esses sistemas com outras formas unidimensionais de suporte ao aprendizado \cite{hobert2019}. Outro aspecto que o suporte metodológico pode oferecer para a comunidade de interesse é valorizar a importância de estudos que debatem a avaliação de sistemas conversacionais no âmbito educacional. Até o momento, dentre as diferentes análises bibliográficas e estudos secundários realizados \cite{io2017, bernardini2018, kuyven2018, roos2018, SMUTNY2020103862, Paschoal:2020FIE}, somente os estudos de \cite{winkler2018} e \citeonline{hobert2019} se preocuparam em discutir sobre as avaliações que são realizadas no contexto de agentes conversacionais na educação. Ambos estudos são utilizados para abordar a problemática existente em avaliações de agentes conversacionais na educação. 

Por fim, vale esclarecer o porquê das abordagens existentes não serem suficientes para resolver a problemática. Conforme mencionado, a disciplina de Engenharia de Software possui processos e metodologias estabelecidas, que podem oferecer apoio aos pesquisadores de outras áreas. No entanto, os processos não conseguem atender uma comunidade formada por engenheiros, matemáticos, pedagogos, psicólogos, artistas, historiadores, geógrafos, físicos, enfermeiros, nutricionistas, médicos, desenvolvedores, etc, que se comunica academicamente com diferentes dialetos. Ainda, as abordagens de Engenharia de Software indicam as atividades que precisam ser realizadas, com exemplos focados em tarefas do ciclo de vida do software e métricas de software. Portanto, não satisfazem pesquisadores que estão tentando compreender efeitos provados pelos agentes conversacionais em contextos de aprendizagem. Por exemplo, se um pesquisador tem interesse em estudar os estímulos motivacionais provocados por um agente, (i) quais variáveis precisará controlar? (ii) quais métricas e instrumentos que podem ser utilizados? (iii) quais ameaças precisam ser mitigadas?. Ademais, as abordagens existentes não reúnem um conjunto essencial de variáveis que precisam ser investigadas ao se avaliar agentes conversacionais. 

Diante dos apontamentos teóricos e questões apresentadas e discutidas ao longo desta seção sobre o tópico de pesquisa, a questão principal de pesquisa deste projeto consiste em:

\begin{cframed}
\centering
\footnotesize
{
\noindent \textit{é possível oferecer apoio efetivo aos pesquisadores que estudam sistemas conversacionais educacionais a conduzir estudos experimentais de modo sistematizado por meio de um framework metodológico?
}
}
\end{cframed}


A partir da discussão sobre a natureza do problema, a hipótese de pesquisa é apresentada: 

\begin{cframed}
\centering
\footnotesize
{
\noindent \textit{um framework metodológico para estudos experimentais pode colaborar na condução de avaliações de agentes conversacionais educacionais, realizadas por pesquisadores neste domínio de pesquisa.
%um framework metodológico ajudará os pesquisadores de agentes conversacionais no domínio educacional a projetar suas avaliações para conseguir obter evidências significativas sobre o seu impacto em  situações e contextos de aprendizagem
}
}
\end{cframed}

%um framework metodológico para estudos experimentais pode colaborar na condução de avaliações de agentes conversacionais educacionais, realizadas por pesquisadores neste domínio de pesquisa. 




\section{Objetivo}

Frente ao contexto, as motivações e a natureza do problema de pesquisa, este projeto de doutorado tem como objetivo apoiar a sistematização de avaliações feitas no contexto de sistemas conversacionais estabelecidos com propósitos educacionais. Esse apoio deve possibilitar que pesquisadores projetem estudos experimentais para avaliar sistemas conversacionais educacionais, cobrindo aspectos técnicos, pedagógicos e identificando possíveis ameaças à validade relacionadas ao estudo. Para isso, este projeto irá investigar a definição de um framework metodológico para apoiar o desenvolvimento dos estudos experimentais. %Esse apoio deve garantir que pesquisadores consigam projetar estudos experimentais sobre agentes com propósitos educacionais, cobrindo aspectos técnicos e pedagógicos e identificar possíveis ameaças à validade que persistem o estudo. Para tanto, o estudo prevê a definição e concepção de um framework metodológico que oferecerá direcionamentos aos pesquisadores. 



%sugestão:
%Esse apoio deve possibilitar que pesquisadores projetem estudos experimentais para avaliar sistemas conversacionais educacionais, cobrindo aspectos técnicos, pedagógicos e identificando possíveis ameaças à validade relacionadas ao estudo. Para isso, este projeto irá investigar a definição de um framework metodológico para apoiar o desenvolvimento dos estudos experimentais. 

Como desdobramento do objetivo principal, pode-se pontuar os seguintes objetivos específicos:

 \begin{itemize}
\item oferecer para os pesquisadores da temática um vocabulário que possibilite a comunidade elaborar e reportar estudos experimentais utilizando a mesma terminologia;
\item disponibilizar um corpo de evidências com indicações de variáveis e um conjunto de métricas (para essas variáveis) para serem investigadas ao longo dos estudos experimentais, possibilitando que a comunidade projete avaliações selecionando as variáveis recomendadas e as métricas apropriadas;
\item conceder diretrizes para a comunidade sobre ameaças que emergem ao se produzir um estudo experimental e instruções para mitigar tais ameaças, de modo que os pesquisadores da temática consigam garantir uma maior validade dos seus estudos experimentais;
\item apresentar orientações para a comunidade planejar, operar, conduzir e reportar os estudos experimentais, possibilitando que os pesquisadores da temática consigam avaliar, comparar, estender e reproduzir estudos experimentais;
\item disponibilizar um mecanismo do tipo agente conversacional, que possibilite a resolução de dúvidas que podem surgir durante o processo experimental de agentes conversacionais educacionais.

 \end{itemize}


\section{Organização do projeto}

Considerando a contextualização do projeto, o restante deste documento foi organizado de modo a colaborar com o desenvolvimento de contribuições, destacar a relevância do projeto para a temática de sistemas conversacionais pedagógicos e situar o leitor sobre o caminhar da pesquisa ao longo do período que o autor está envolvido com o projeto. Nesse sentido, os próximos capítulo deste documento de qualificação contemplam uma coletânea de estudos, em formato de artigos científicos, cada um destacado a seguir:

 \begin{itemize}

    \item o segundo capítulo deste projeto é constituído por um artigo que analisa pesquisas conduzidas no âmbito de formações em nível de mestrado e doutorado sobre agentes conversacionais com propósitos educacionais. Na análise, demonstra-se a fragilidade da sistematização e seleção de variáveis adotadas para avaliar os agentes, em projetos que geram contribuições relevantes para a comunidade;

    \item o terceiro capítulo deste projeto é constituído pelo artigo que visa apresentar o conceito de agente conversacional, o processo de desenvolvimento do agente TOB--STT, suas funcionalidades e um estudo de viabilidade que analisa a qualidade da resposta oferecida pelo agente para estudantes de teste de software;
 
    \item o quarto capítulo deste projeto é constituído por um artigo que envolve uma avaliação experimental do suporte educacional oferecido pelo agente TOB--STT em uma situação de aprendizagem que o aluno não tem contato com professor para resolver suas dúvidas. O estudo aborda também desdobramentos necessários para o estudo sobre o papel e impacto de agentes conversacionais no domínio de ensino de teste de software;

    \item por fim, o quinto capítulo, que segue uma estrutura de estudos apresentados no âmbito de simpósio doutoral e workshop de teses de doutorado, contém a proposta da pesquisa que inclui métodos que serão utilizados na pesquisa, atividades e contribuições esperadas.
    
 \end{itemize}
 
Vale salientar que os artigos científicos que constituem o projeto além de contribuírem com o embasamento teórico-prático da proposta, servem  para indicar que o autor tem a expertise necessária para conduzir esta pesquisa.



